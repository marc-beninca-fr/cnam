\section{Introduction}

À notre époque dans laquelle l’informatique est devenue omniprésente,\\
la consultation d’informations numérisées n’a jamais été aussi importante.

Ainsi devient-il de plus en plus nécessaire de mener diverses réflexions,
pour tendre vers une production toujours plus fiable de documents,
afin de pouvoir véhiculer au mieux les informations numérisées.

Plusieurs supports numériques sont apparus au fil du temps :
\begin{itmz}
\item{le texte}
\item{l’image}
\item{l’audio}
\item{la vidéo}
\end{itmz}
La forme la plus courante de transmission d’informations restant aujourd’hui\\
le document, combinant à la fois des contenus textuels et imagés,\\
le plus souvent encapsulés dans un fichier \gls{pdf}.

Deux grandes parties seront ici abordées à son sujet :
\begin{enum}
\item{quels objectifs se fixer pour améliorer la production de documents ?}
\item{vers quels moyens se tourner pour tendre vers de tels objectifs ?}
\end{enum}

\pagebreak
