\section{Conclusion}

Il est aujourd’hui possible de mettre en place\\
une véritable industrialisation de la production de documents, en combinant :
\begin{itmz}
\item{l’utilisation de logiciels libres}
\item{le choix de formats ouverts et textuels}
\item{une gestion de configuration distribuée}
\item{l’automatisation des tâches de fabrication}
\item{l’adoption d’une démarche \gls{wysiwym}}
\item{l’utilisation de la signature numérique}
\item{un déploiement automatique multi-sites}
\end{itmz}

Cette rationalisation permet de réduire drastiquement les risques, en\\
renforçant des points essentiels de la \gls{ssi} :
\begin{itmz}
\item{authentification}
\item{intégrité}
\item{disponibilité}
\end{itmz}
La confidentialité via \gls{gpg} \cite{gpg} n’est ici pas abordée, car ce sujet
aurait occupé une place disproportionnée par rapport aux autres parties.

\hr

Ce document applique lui-même toutes les recommandations qu’il préconise.

Suivent en annexes quelques fichiers exemples du processus mis en place…

\pagebreak
