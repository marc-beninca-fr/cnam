\ml
{\chapter{Proposal: an incremental live workflow}}
{\chapter{Proposition : un fonctionnement autonome incrémental}}

\ml{Pros}{Avantages}:
\begin{itmz}
\item{\ml{reboot = restore}
{redémarrage = restauration}}
\item{\ml{update = backup}
{mise à jour = sauvegarde}}
\item{\ml{separation of system and data}
{séparation système et données}}
\end{itmz}

\ml{Cons}{Inconvénients}:
\begin{itmz}
\item{\ml{exhaustive manual procedure}
{maintenance manuelle exhaustive}}
\end{itmz}

\ml
{\section{Implement the workflow}}
{\section{Mettre en œuvre le fonctionnement}}

\ml
{\subsection{Mirror official and third-party repositories}}
{\subsection{Cloner des dépôts officiels et d’éditeurs tiers}}

\ml{Pros}{Avantages}:
\begin{itmz}
\item{\ml{\todo}
{\todo}}
\end{itmz}

\ml{Cons}{Inconvénients}:
\begin{itmz}
\item{\ml{\todo}
{\todo}}
\end{itmz}

\ml
{\subsubsection{Synchronize local mirrors}}
{\subsubsection{Synchroniser des miroirs locaux}}

\paragraph{apt-mirror}

\begin{itmz}
\item{\ml{translations}
{traductions} (Translation-*.bz2)}
\item{\ml{architecture independant contents}
{contenus indépendants de l’architecture} (Contents-all.gz)}
\item{InRelease \ml{with some third-party repositories}
{avec certains dépôts éditeurs}}
\end{itmz}

\paragraph{debmirror}

\paragraph{ftpsync}

\ml
{\subsubsection{Select useful architectures}}
{\subsubsection{Choisir les architectures utiles}}

\paragraph{amd64}

\paragraph{arm64}

\paragraph{armhf}

\paragraph{i386}

\ml
{\subsubsection{Check integrity}}
{\subsubsection{Vérifier l’intégrité}}

\ml{Pros}{Avantages}:
\begin{itmz}
\item{\ml{avoid errors during future package installations}
{éviter des erreurs lors de futures installations de paquets}}
\end{itmz}

\ml{Cons}{Inconvénients}:
\begin{itmz}
\item{\ml{no tool exists}
{aucun outil n’existe}}
\end{itmz}

\ml
{\subsection{Build a live file system}}
{\subsection{Construire un système de fichiers autonome}}

\paragraph{Debian GNU/Linux}

\ml
{\subsubsection{Install specific packages}}
{\subsubsection{Installer les paquets spécifiques}}

\paragraph{\ml{Bare metal}{Machine physique}}

\paragraph{\ml{Virtual machine}{Machine virtuelle}}

\paragraph{\ml{Container}{Conteneur}}

\ml
{\subsubsection{Create a minimal file system base}}
{\subsubsection{Créer un système de fichiers minimal}}

\paragraph{debootstrap}

\ml
{\subsubsection{Turn a system into a systems factory}}
{\subsubsection{Équiper un système pour en fabriquer d’autres}}

\ml
{\subsubsection{Turn a file system into a live one}}
{\subsubsection{Rendre un système de fichiers autonome}}

\paragraph{live-boot}

\paragraph{update-initramfs}

\ml
{\subsubsection{Install additional packages}}
{\subsubsection{Installer des paquets supplémentaires}}

\ml
{\subsubsection{Link specific data to persistent storage}}
{\subsubsection{Lier certaines données à du stockage persistant}}

\ml
{\subsection{Encapsulate in an image file}}
{\subsection{Encapsuler dans un fichier image}}

\ml
{\subsubsection{Use a format suited for read-only mounting}}
{\subsubsection{Utiliser un format adapté au montage en lecture}}

\paragraph{SquashFS}

\ml
{\subsubsection{Choose a compression algorithm}}
{\subsubsection{Choisir un algorithme de compression}}

\paragraph{gzip}

\paragraph{lzma}

\paragraph{lzo}

\paragraph{lz4}

\paragraph{xz}

\paragraph{zstd}

\ml
{\subsubsection{Encapsulate in a hybrid image file}}
{\subsubsection{Encapsuler dans un fichier image hybride}}

\paragraph{ISO}

\ml
{\subsection{Secure the produced image file}}
{\subsection{Sécuriser un fichier image produit}}

\ml
{\subsubsection{Compute an integrity checksum}}
{\subsubsection{Calculer une somme de contrôle d’intégrité}}

\paragraph{SHA-256}

\paragraph{SHA-512}

\ml
{\subsubsection{Sign to certify authenticity}}
{\subsubsection{Signer pour certifier l’authenticité}}

\ml
{\subsection{Boot secure image files}}
{\subsection{Amorcer des fichiers images sécurisés}}

\ml
{\subsubsection{Create standalone boot images}}
{\subsubsection{Créer des images de démarrage}}

\paragraph{GRUB}

\paragraph{BIOS}

\paragraph{UEFI}

\paragraph{Secure boot}

\ml
{\subsubsection{Create a boot menu}}
{\subsubsection{Créer un menu de démarrage}}

\paragraph{grub.cfg}

\paragraph{loopback}

\paragraph{squash4}

\paragraph{iso9660}

\ml
{\subsubsection{Check integrity and authenticity}}
{\subsubsection{Vérifier intégrité et authentiticté}}

\paragraph{gcry\_sha256}

\paragraph{gcry\_sha512}

\paragraph{pgp}

\ml
{\subsubsection{Load images in random access memory}}
{\subsubsection{Charger des images en mémoire vive}}

\paragraph{overlayfs}

\ml
{\subsection{Incremental updating}}
{\subsection{Mise à niveau incrémentale}}

\ml
{\subsubsection{Create a new image file}}
{\subsubsection{Fabriquer une nouvelle image}}

\ml
{\subsubsection{Avoid an unnecessary reboot}}
{\subsubsection{Éviter un redémarrage facultatif}}

\ml{Pros}{Avantages}:
\begin{itmz}
\item{\ml{no down time}
{aucune indisponibilité}}
\item{\ml{just replay the modifications on the system in memory}
{répéter simplement les modifications sur le système en mémoire}}
\end{itmz}

\ml{Cons}{Inconvénients}:
\begin{itmz}
\item{\ml{\todo}
{\todo}}
\end{itmz}

\ml
{\subsubsection{Reduce the duration of a mandatory reboot}}
{\subsubsection{Réduire la durée d’un redémarrage obligatoire}}

\paragraph{kexec-tools}

\ml
{\section{Automate the workflow}}
{\section{Automatiser le fonctionnement}}

\ml
{\subsection{Check integrity of local repositories}}
{\subsection{Vérifier l’intégrité des dépôts locaux}}

\ml
{\subsection{Build complete live file systems from scratch}}
{\subsection{Construire des systèmes de fichiers autonomes complets}}

\ml
{\subsection{Create new files by updating existing images}}
{\subsection{Créer de nouveaux fichiers par mise à jour d’images}}

\ml
{\subsection{Generate a boot menu on-the-fly}}
{\subsection{Générer un menu de démarrage à la volée}}
