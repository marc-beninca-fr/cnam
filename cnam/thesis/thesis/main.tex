% set default size and document class
\documentclass[12pt]{report}

% set default size and document class
\documentclass[12pt]{memoir}

% set fonts
\usepackage{fontspec}
\setmainfont{DejaVu Sans}
\setmonofont{DejaVu Sans Mono}

% set languages
\usepackage{polyglossia}
\setmainlanguage{\mainlanguage}
\setotherlanguages{\otherlanguages}

% handle multiple languages

\newcommand{\ifstr}[4]{\expandafter\ifstrequal\expandafter{#1}{#2}{#3}{#4}}

\newcommand{\warn}[1]{/!\textbackslash #1 /!\textbackslash}

\newcommand{\ml}[2]{%
\ifstr{\mainlanguage}{en}{\ifstrempty{#1}{\warn{TRANSLATE ME}}{#1}}{%
\ifstr{\mainlanguage}{fr}{\ifstrempty{#2}{\warn{TRADUISEZ-MOI}}{#2}}{%
SHOULD\_NOT\_HAPPEN!%
}}%
}

\newcommand{\en}[1]{\textenglish{#1}}

\newcommand{\fr}[1]{\textfrench{#1}}


\singlespacing

\begin{center}
{\bfseries
\begin{large}

{\larger[2] C}ONSERVATOIRE {\larger[2] N}ATIONAL
DES {\larger[2] A}RTS ET {\larger[2] M}ÉTIERS

\ml
{ASSOCIATE REGIONAL CENTER OF NEW-AQUITAINE}
{CENTRE RÉGIONAL ASSOCIÉ DE NOUVELLE-AQUITAINE}

\hrq

\ml{THESIS}{MÉMOIRE}

\ml{defended in order to obtain}{présenté en vue d’obtenir}

\ml{the CNAM ENGINEER DIPLOMA}{le DIPLÔME d’INGÉNIEUR CNAM}

\ml{SPECIALIZATION: Computer Science}{SPÉCIALITÉ : Informatique}

\ml
{OPTION: Networks, Systems and Multimedia}
{OPTION : Réseaux, Systèmes et Multimédia}

\vfill
\ml{by}{par}
\vfill

\author

\hrq

\title

\ml
{Defended on Month DD, 2021}
{Soutenu le JJ Mois 2021}

\hrq

JURY

\end{large}
\vspace{1em}
\begin{small}

{\renewcommand{\arraystretch}{1.5} \begin{tabular}[t]{rllll}
PRÉSIDENT : & Civilité & Prénom & NOM & \textit{\renewcommand{\arraystretch}{1} \begin{tabular}[t]{@{}l@{}}Fonction\\Organisme\end{tabular}} \\
MEMBRES : & Civilité & Prénom & NOM & \textit{\renewcommand{\arraystretch}{1} \begin{tabular}[t]{@{}l@{}}Fonction\\Organisme\end{tabular}} \\
& Civilité & Prénom & NOM & \textit{\renewcommand{\arraystretch}{1} \begin{tabular}[t]{@{}l@{}}Fonction\\Organisme\end{tabular}}
\end{tabular}}

\end{small}
}
\end{center}

\thispagestyle{empty}

\pagebreak


\onehalfspacing

\def\thanks{\ml{Acknowledgements}{Remerciements}}
\chapter*{\thanks}
\addcontentsline{toc}{chapter}{\thanks}

\printglossary[type=\acronymtype,
title=\ml{Acronyms}{Abréviations}]

\newacronym{os}
{\ml{OS}{SE}}
{\ml{Operating System}{Système d’Exploitation}}

\pagebreak

\ml
{\def\gloss{Glossary}}
{\def\gloss{Glossaire}}
\printglossary
[title=\gloss]

\pagebreak

\renewcommand{\contentsname}{\ml{Contents}{Plan}}

\renewcommand{\cftchapleader}{\cftdotfill{\cftdotsep}}

\tableofcontents

\section{Introduction}
%–––––––––––––––––––––––––––––––––––––––––––––––––––––––––––––––––––––––––––––––
\subsection{Machine learning}
%–––––––––––––––––––––––––––––––––––––––––––––––––––––––––––––––––––––––––––––––
\subsection{Méthodes}
%⋅⋅⋅⋅⋅⋅⋅⋅⋅⋅⋅⋅⋅⋅⋅⋅⋅⋅⋅⋅⋅⋅⋅⋅⋅⋅⋅⋅⋅⋅⋅⋅⋅⋅⋅⋅⋅⋅⋅⋅⋅⋅⋅⋅⋅⋅⋅⋅⋅⋅⋅⋅⋅⋅⋅⋅⋅⋅⋅⋅⋅⋅⋅⋅⋅⋅⋅⋅⋅⋅⋅⋅⋅⋅⋅⋅⋅⋅⋅
\begin{frame}{\bititle\\Introduction / Méthodes non supervisées}

\end{frame}
%⋅⋅⋅⋅⋅⋅⋅⋅⋅⋅⋅⋅⋅⋅⋅⋅⋅⋅⋅⋅⋅⋅⋅⋅⋅⋅⋅⋅⋅⋅⋅⋅⋅⋅⋅⋅⋅⋅⋅⋅⋅⋅⋅⋅⋅⋅⋅⋅⋅⋅⋅⋅⋅⋅⋅⋅⋅⋅⋅⋅⋅⋅⋅⋅⋅⋅⋅⋅⋅⋅⋅⋅⋅⋅⋅⋅⋅⋅⋅
\begin{frame}{\bititle\\Introduction / Méthodes semi-supervisées}

\end{frame}
%⋅⋅⋅⋅⋅⋅⋅⋅⋅⋅⋅⋅⋅⋅⋅⋅⋅⋅⋅⋅⋅⋅⋅⋅⋅⋅⋅⋅⋅⋅⋅⋅⋅⋅⋅⋅⋅⋅⋅⋅⋅⋅⋅⋅⋅⋅⋅⋅⋅⋅⋅⋅⋅⋅⋅⋅⋅⋅⋅⋅⋅⋅⋅⋅⋅⋅⋅⋅⋅⋅⋅⋅⋅⋅⋅⋅⋅⋅⋅
\begin{frame}{\bititle\\Introduction / Méthodes supervisées}

\end{frame}


\ml
{\chapter{Problem: maintenance of operating systems}}
{\chapter{Problématique : maintenance des systèmes d’exploitation}}

\begin{itmz}
\item{\ml{updates}
{mises à jour}}
\item{\ml{unavailability}
{indisponibilité}}
\item{\ml{backups policy}
{politique de sauvegardes}}
\item{\ml{testing backups}
{tests des sauvegardes}}
\item{\ml{snapshots}
{instantanés}}
\item{\ml{restorations}
{restaurations}}
\item{\ml{configuration recipes}
{recettes de configuration}}
\end{itmz}

\ml
{\section{File systems, installed on partitions, with write access}}
{\section{Systèmes de fichiers, installés sur partitions, avec accès en écriture}}

\ml
{\subsection{Conventional file systems}}
{\subsection{Système de fichiers conventionnel}}

\begin{itmz}
\item{ext2}
\item{ext3}
\item{ext4}
\item{jfs}
\item{xfs}
\end{itmz}

\ml{Pros}{Avantages}:
\begin{itmz}
\item{\ml{\todo}
{\todo}}
\end{itmz}

\ml{Cons}{Inconvénients}:
\begin{itmz}
\item{\ml{\todo}
{\todo}}
\end{itmz}

\ml
{\subsection{File systems managed with configuration recipes}}
{\subsection{Système de fichiers géré par des recettes configuration}}

\begin{itmz}
\item{ansible}
\item{chef}
\item{puppet}
\end{itmz}

\ml{Pros}{Avantages}:
\begin{itmz}
\item{\ml{\todo}
{\todo}}
\end{itmz}

\ml{Cons}{Inconvénients}:
\begin{itmz}
\item{\ml{\todo}
{\todo}}
\end{itmz}

\ml
{\subsection{File systems supporting snapshots}}
{\subsection{Système de fichiers avec gestion d’instantanés}}

\begin{itmz}
\item{btrfs}
\item{zfs}
\end{itmz}

\ml{Pros}{Avantages}:
\begin{itmz}
\item{\ml{\todo}
{\todo}}
\end{itmz}

\ml{Cons}{Inconvénients}:
\begin{itmz}
\item{\ml{\todo}
{\todo}}
\end{itmz}

\ml
{\section{Live images, installationless, with read access}}
{\section{Images autonomes, sans installation, avec accès en lecture seule}}

\ml
{\subsection{Boot without persistence}}
{\subsection{Amorçage sans gestion de persistance}}

\ml{Pros}{Avantages}:
\begin{itmz}
\item{\ml{\todo}
{\todo}}
\end{itmz}

\ml{Cons}{Inconvénients}:
\begin{itmz}
\item{\ml{\todo}
{\todo}}
\end{itmz}

\ml
{\subsection{Boot with persistence}}
{\subsection{Amorçage avec gestion de persistance}}

\ml{Pros}{Avantages}:
\begin{itmz}
\item{\ml{\todo}
{\todo}}
\end{itmz}

\ml{Cons}{Inconvénients}:
\begin{itmz}
\item{\ml{\todo}
{\todo}}
\end{itmz}

\ml
{\section{Existing alternatives}}
{\section{Alternatives existantes}}

\ml
{\chapter{Proposal: an incremental live workflow}}
{\chapter{Proposition : un fonctionnement autonome incrémental}}

\ml{Pros}{Avantages}:
\begin{itmz}
\item{\ml{reboot = restore}
{redémarrage = restauration}}
\item{\ml{update = backup}
{mise à jour = sauvegarde}}
\item{\ml{separation of system and data}
{séparation système et données}}
\end{itmz}

\ml{Cons}{Inconvénients}:
\begin{itmz}
\item{\ml{exhaustive manual procedure}
{maintenance manuelle exhaustive}}
\end{itmz}

\ml
{\section{Implement the workflow}}
{\section{Mettre en œuvre le fonctionnement}}

\ml
{\subsection{Mirrors of official and third-party repositories}}
{\subsection{Miroirs de dépôts officiels et éditeurs tiers}}

\ml{Pros}{Avantages}:
\begin{itmz}
\item{\ml{\todo}
{\todo}}
\end{itmz}

\ml{Cons}{Inconvénients}:
\begin{itmz}
\item{\ml{\todo}
{\todo}}
\end{itmz}

\ml
{\subsubsection{Local synchronization}}
{\subsubsection{Synchronisation locale}}

\paragraph{apt-mirror}

\begin{itmz}
\item{\ml{translations}
{traductions} (Translation-*.bz2)}
\item{\ml{architecture independant contents}
{contenus indépendants de l’architecture} (Contents-all.gz)}
\item{InRelease \ml{with some third-party repositories}
{avec certains dépôts éditeurs}}
\end{itmz}

\paragraph{debmirror}

\paragraph{ftpsync}

\ml
{\subsubsection{Select useful architectures}}
{\subsubsection{Choisir les architectures utiles}}

\paragraph{amd64}

\paragraph{arm64}

\paragraph{armhf}

\paragraph{i386}

\ml
{\subsubsection{Check integrity}}
{\subsubsection{Vérifier l’intégrité}}

\ml{Pros}{Avantages}:
\begin{itmz}
\item{\ml{avoid errors during future package installations}
{éviter des erreurs lors de futures installations de paquets}}
\end{itmz}

\ml{Cons}{Inconvénients}:
\begin{itmz}
\item{\ml{no tool exists}
{aucun outil n’existe}}
\end{itmz}

\ml
{\subsection{Building a live file system}}
{\subsection{Construction d’un système de fichiers autonome}}

\paragraph{Debian GNU/Linux}

\ml
{\subsubsection{Install specific packages}}
{\subsubsection{Installer les paquets spécifiques}}

\paragraph{\ml{Bare metal}{Machine physique}}

\paragraph{\ml{Virtual machine}{Machine virtuelle}}

\paragraph{\ml{Container}{Conteneur}}

\ml
{\subsubsection{Create a minimal file system base}}
{\subsubsection{Créer un système de fichiers minimal}}

\paragraph{debootstrap}

\ml
{\subsubsection{Turn a system into a systems factory}}
{\subsubsection{Équiper un système pour en fabriquer d’autres}}

\ml
{\subsubsection{Turn a file system into a live one}}
{\subsubsection{Rendre un système de fichiers autonome}}

\paragraph{live-boot}

\paragraph{update-initramfs}

\ml
{\subsubsection{Install additional packages}}
{\subsubsection{Installer des paquets supplémentaires}}

\ml
{\subsubsection{Data persistence through symbolic links}}
{\subsubsection{Déport des données persistantes avec liens symboliques}}

\ml
{\subsection{Encapsulate in an image file}}
{\subsection{Encapsuler dans un fichier image}}

\ml
{\subsubsection{Use a format suited for read-only mounting}}
{\subsubsection{Utiliser un format adapté au montage en lecture}}

\paragraph{SquashFS}

\ml
{\subsubsection{Choose a compression algorithm}}
{\subsubsection{Choisir un algorithme de compression}}

\paragraph{gzip}

\paragraph{lzma}

\paragraph{lzo}

\paragraph{lz4}

\paragraph{xz}

\paragraph{zstd}

\ml
{\subsubsection{Encapsulate in a hybrid image file}}
{\subsubsection{Encapsuler dans un fichier image hybride}}

\paragraph{ISO}

\ml
{\subsection{Secure the image file produced}}
{\subsection{Sécuriser un fichier image produit}}

\ml
{\subsubsection{Compute an integrity checksum}}
{\subsubsection{Calculer une somme de contrôle d’intégrité}}

\paragraph{SHA-256}

\paragraph{SHA-512}

\ml
{\subsubsection{Sign to certify authenticity}}
{\subsubsection{Signer pour certifier l’authenticité}}

\ml
{\subsection{Boot secure image files}}
{\subsection{Amorcer des fichiers images sécurisés}}

\ml
{\subsubsection{Create standalone boot images}}
{\subsubsection{Créer des images de démarrage}}

\paragraph{GRUB}

\paragraph{BIOS}

\paragraph{UEFI}

\paragraph{Secure boot}

\ml
{\subsubsection{Create a boot menu}}
{\subsubsection{Créer un menu de démarrage}}

\paragraph{grub.cfg}

\paragraph{loopback}

\paragraph{squash4}

\paragraph{iso9660}

\ml
{\subsubsection{Check integrity and authenticity}}
{\subsubsection{Vérifier intégrité et authentiticté}}

\paragraph{gcry\_sha256}

\paragraph{gcry\_sha512}

\paragraph{pgp}

\ml
{\subsubsection{Load images in random access memory}}
{\subsubsection{Charger des images en mémoire vive}}

\paragraph{overlayfs}

\ml
{\subsection{Incremental updating}}
{\subsection{Mise à niveau incrémentale}}

\ml
{\subsubsection{Create a new image file}}
{\subsubsection{Fabriquer une nouvelle image}}

\ml
{\subsubsection{Avoid an unnecessary reboot}}
{\subsubsection{Éviter un redémarrage facultatif}}

\ml{Pros}{Avantages}:
\begin{itmz}
\item{\ml{no down time}
{aucune indisponibilité}}
\item{\ml{just replay the modifications on the system in memory}
{répéter simplement les modifications sur le système en mémoire}}
\end{itmz}

\ml{Cons}{Inconvénients}:
\begin{itmz}
\item{\ml{\todo}
{\todo}}
\end{itmz}

\ml
{\subsubsection{Reduce the duration of a mandatory reboot}}
{\subsubsection{Réduire la durée d’un redémarrage obligatoire}}

\paragraph{kexec-tools}

\ml
{\section{Automate the workflow}}
{\section{Automatiser le fonctionnement}}

\ml
{\subsection{Check integrity of local repositories}}
{\subsection{Vérifier l’intégrité des dépôts locaux}}

\ml
{\subsection{Build complete live file systems from scratch}}
{\subsection{Construire des systèmes de fichiers autonomes complets}}

\ml
{\subsection{Create new files by updating existing images}}
{\subsection{Créer de nouveaux fichiers par mise à jour d’images}}

\ml
{\subsection{Generate a boot menu on-the-fly}}
{\subsection{Générer un menu de démarrage à la volée}}

\ml
{\chapter{Results}}
{\chapter{Résultats}}

%–––––––––––––––––––––––––––––––––––––––––––––––––––––––––––––––––––––––––––––––
\section{Conclusion}
%⋅⋅⋅⋅⋅⋅⋅⋅⋅⋅⋅⋅⋅⋅⋅⋅⋅⋅⋅⋅⋅⋅⋅⋅⋅⋅⋅⋅⋅⋅⋅⋅⋅⋅⋅⋅⋅⋅⋅⋅⋅⋅⋅⋅⋅⋅⋅⋅⋅⋅⋅⋅⋅⋅⋅⋅⋅⋅⋅⋅⋅⋅⋅⋅⋅⋅⋅⋅⋅⋅⋅⋅⋅⋅⋅⋅⋅⋅⋅
\begin{frame}{\bititle\\Conclusion}
\begin{columns}\begin{column}{.5\textwidth}

\begin{itemize}
\item<1-> polyvalence
\item<2-> flexibilité
\item<3-> variété
\item<4-> disponibilité
\item<5-> évolution
\end{itemize}

\end{column}\begin{column}{.5\textwidth}

\only<1>{\imgh{0}{15}{knife}}
\only<2>{\imgh{0}{13}{flexible}}
\only<3>{\imgh{0}{10}{workflow}}
\only<4>{\imgh{0}{11.5}{tools}}
\only<5>{\imgh{0}{9}{evolution}}

\end{column}\end{columns}
\end{frame}


\appendix

\ml
{\chapter{Appendix}}
{\chapter{Annexe}}
\pagebreak

\ml
{\def\bib{References}}
{\def\bib{Références}}
\printbibliography[heading=bibintoc,
title=\bib]

\pagebreak

\def\figures{Figures}
\renewcommand{\listfigurename}{\figures}
\listoffigures
\addcontentsline{toc}{chapter}{\figures}

\renewcommand{\listtablename}{\ml{Tables}{Tableaux}}
\listoftables


\singlespacing

\ml
{\def\back{Summaries}}
{\def\back{Résumés}}
\chapter*{}
\addcontentsline{toc}{chapter}{\back}

\begin{center}\begin{large}\textbf{
\titlelong\\
\titlesub
}\end{large}\end{center}

\typelong,\\
Bordeaux \dateshort.

\summaries

\thispagestyle{empty}

\pagebreak


\end{document}
