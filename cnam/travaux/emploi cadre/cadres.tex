\documentclass[20pt]{extarticle}
%–––––––––––––––––––––––––––––––––––––––––––––––––––––––––––––––––––––––––––––––
\usepackage{extsizes}
\usepackage{fontspec}
\usepackage[a4paper,landscape,
bmargin=10mm,lmargin=20mm,rmargin=20mm,tmargin=25mm]{geometry}
\usepackage{metalogo}
\usepackage{tocloft}
\usepackage{wallpaper}
%–––––––––––––––––––––––––––––––––––––––––––––––––––––––––––––––––––––––––––––––
\setcounter{secnumdepth}{1}
\setmainfont{DejaVu Sans}
\pagenumbering{gobble}
%–––––––––––––––––––––––––––––––––––––––––––––––––––––––––––––––––––––––––––––––
\begin{document}
%–––––––––––––––––––––––––––––––––––––––––––––––––––––––––––––––––––––––––––––––
\ThisURCornerWallPaper{.25}{../cnam.png}
\title{Perspectives de\newline
l’emploi cadre 2018}
\author{Marc BENINCA}
\date{26 / 02 / 2020}
\maketitle
\pagebreak
%–––––––––––––––––––––––––––––––––––––––––––––––––––––––––––––––––––––––––––––––
\ThisURCornerWallPaper{.25}{../cnam.png}
Perspectives de l’emploi cadre 2018
\renewcommand{\contentsname}{Sommaire}
\cftsetindents{section}{1em}{1.5em}
\cftsetindents{subsection}{2.5em}{2.5em}
\renewcommand{\cftsecleader}{\hfill}
\renewcommand{\cftsubsecleader}{\hfill}
\tableofcontents
\pagebreak
%–––––––––––––––––––––––––––––––––––––––––––––––––––––––––––––––––––––––––––––––
\section{Source}
\subsection{APEC}
\ThisLRCornerWallPaper{.83}{apec.png}
\pagebreak
%–––––––––––––––––––––––––––––––––––––––––––––––––––––––––––––––––––––––––––––––
\section{Traitements}
\subsection{Calculs}
\begin{itemize}
\item{estimations moyennes}
\item{totaux de recrutements}
\item{pourcentages de croissance des volumes}
\item{pourcentages de répartition par secteurs}
\item{évolution des pourcentages de répartition}
\end{itemize}
\subsection{Analyse}
\begin{itemize}
\item{tri des différents secteurs par volumes décroissants}
\item{identification des évolutions en perspective}
\item{représentations graphiques des données disponibles}
\end{itemize}
\pagebreak
%–––––––––––––––––––––––––––––––––––––––––––––––––––––––––––––––––––––––––––––––
\section{Chiffres}
\ThisLRCornerWallPaper{1}{croissances.png}
\begin{itemize}
\item{volumes}
\item{répartition}
\end{itemize}
\pagebreak
%–––––––––––––––––––––––––––––––––––––––––––––––––––––––––––––––––––––––––––––––
\subsection{Répartition}
\begin{itemize}
\item{↗ 2 secteurs augmenteront leur ratio\newline
dans la répartition des recrutements :
    \begin{itemize}
    \item{+1,7\% pour « Informatique »}
    \item{+0,7\% pour « Études, Recherche et Développement »}
    \end{itemize}}
\item{parmi les 3 secteurs qui se partageront\newline
toujours plus de la moitié des recrutements :
    \begin{itemize}
    \item{20,0→21,7\% pour « Informatique »}
    \item{18,0→18,7\% pour « Études, Recherche et Développement »}
    \item{18,0→17,3\% pour « Commercial, Marketing »}
    \end{itemize}}
\item{→ 1 seul secteur devrait maintenir sa proportion :
    \begin{itemize}
    \item{6,4\% « Achats, qualité, maintenance, logistique, sécurité »}
    \end{itemize}}
\item{↘ les 5 autres verront leur part diminuer}
\end{itemize}
\pagebreak
%⋅⋅⋅⋅⋅⋅⋅⋅⋅⋅⋅⋅⋅⋅⋅⋅⋅⋅⋅⋅⋅⋅⋅⋅⋅⋅⋅⋅⋅⋅⋅⋅⋅⋅⋅⋅⋅⋅⋅⋅⋅⋅⋅⋅⋅⋅⋅⋅⋅⋅⋅⋅⋅⋅⋅⋅⋅⋅⋅⋅⋅⋅⋅⋅⋅⋅⋅⋅⋅⋅⋅⋅⋅⋅⋅⋅⋅⋅⋅
\ThisLRCornerWallPaper{1}{répartition.png}
\begin{itemize}
\item{intérieur : 2017}
\item{extérieur : 2018}
\end{itemize}
\pagebreak
%–––––––––––––––––––––––––––––––––––––––––––––––––––––––––––––––––––––––––––––––
\subsection{Volumes}
\begin{itemize}
\item{↗ hausse globale des recrutements cadre de 7,4 \%}
\item{⇗ 2 secteurs en très forte croissance :
    \begin{itemize}
    \item{+16\% pour « Informatique »}
    \item{+12\% pour « Études, Recherche et Développement »}
    \end{itemize}}
\item{↗ 5 autres secteurs en augmentation :
    \begin{itemize}
    \item{+7,2\% pour « Achats, Qualité, Logistique, Sécurité »}
    \item{+6,8\% pour « Finance, comptabilité, audit »}
    \item{+4,7\% pour « Exploitation Tertiaire »}
    \item{+3,3\% pour « Commercial, Marketing »}
    \item{+2,8\% pour « Production industrielle, chantier »}
    \end{itemize}}
\item{↘ 2 secteurs en légère baisse :
    \begin{itemize}
    \item{−3,2\% pour « Admin, RH, Comm, Juridique »}
    \item{−2,5\% pour « Direction générale »}
    \end{itemize}}
\end{itemize}
\pagebreak
%⋅⋅⋅⋅⋅⋅⋅⋅⋅⋅⋅⋅⋅⋅⋅⋅⋅⋅⋅⋅⋅⋅⋅⋅⋅⋅⋅⋅⋅⋅⋅⋅⋅⋅⋅⋅⋅⋅⋅⋅⋅⋅⋅⋅⋅⋅⋅⋅⋅⋅⋅⋅⋅⋅⋅⋅⋅⋅⋅⋅⋅⋅⋅⋅⋅⋅⋅⋅⋅⋅⋅⋅⋅⋅⋅⋅⋅⋅⋅
\ThisLRCornerWallPaper{1}{volumes.png}
 
\pagebreak
%–––––––––––––––––––––––––––––––––––––––––––––––––––––––––––––––––––––––––––––––
\section{Conclusion}
\subsection{Secteurs porteurs}
Les 2 progressant le plus, tant en répartition qu’en volumes :
\begin{itemize}
\item{« Informatique »}
\item{« Études, Recherche et Développement »}
\end{itemize}
\subsection{Raisons d’orientation}
Liés aux technologies actuelles, les 2 grandes transformations :
\begin{itemize}
\item{\textbf{numérique}\newline
analyse, conception, réalisation,\newline
numérisation, traitement, automatisation}
\item{\textbf{digitale}\newline
matériel, application, utilisation,\newline
intégration, connectivité, mobilité}
\end{itemize}
\pagebreak
%–––––––––––––––––––––––––––––––––––––––––––––––––––––––––––––––––––––––––––––––
\end{document}
%–––––––––––––––––––––––––––––––––––––––––––––––––––––––––––––––––––––––––––––––
