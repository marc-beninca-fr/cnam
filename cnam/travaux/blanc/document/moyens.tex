\section{Moyens}

\subsection{Outils logiciels}

\subsubsection{Propriétaires}

\subsubsection{Open source}

\subsubsection{Libres}

\subsection{Formats de données}

\subsubsection{Fermés}

\subsubsection{Obfusqués}

\subsubsection{Ouverts}

\pagebreak
%⋅⋅⋅⋅⋅⋅⋅⋅⋅⋅⋅⋅⋅⋅⋅⋅⋅⋅⋅⋅⋅⋅⋅⋅⋅⋅⋅⋅⋅⋅⋅⋅⋅⋅⋅⋅⋅⋅⋅⋅⋅⋅⋅⋅⋅⋅⋅⋅⋅⋅⋅⋅⋅⋅⋅⋅⋅⋅⋅⋅⋅⋅⋅⋅⋅
\subsection{Types de formats}
Dans un \gls{si} :\\
\gls{wysiwym} \cite{wysiwym}\\
\gls{wysiwyg} \cite{wysiwyg}\\
\gls{wysiwym} est préférable à \gls{wysiwyg}

\subsubsection{Binaire}

\subsubsection{Texte}

\subsection{Rationalisation}

\subsubsection{Gestion de versions}

\subsubsection{Processus de fabrication}

\subsubsection{Logique de sauvegardes}

\subsubsection{Réplication de contenus}

\subsubsection{Signature numérique}

\pagebreak
%⋅⋅⋅⋅⋅⋅⋅⋅⋅⋅⋅⋅⋅⋅⋅⋅⋅⋅⋅⋅⋅⋅⋅⋅⋅⋅⋅⋅⋅⋅⋅⋅⋅⋅⋅⋅⋅⋅⋅⋅⋅⋅⋅⋅⋅⋅⋅⋅⋅⋅⋅⋅⋅⋅⋅⋅⋅⋅⋅⋅⋅⋅⋅⋅⋅
\subsection{Manipulation}

\subsubsection{Découpe de documents}

\subsubsection{Extraction d’images}

\subsubsection{Rotation d’éléments}

\subsubsection{Compression d’images}

\subsubsection{Résolution d’impression}

\subsubsection{Assemblage de documents}

\subsection{Automatisation}

\subsubsection{Fichiers de fabrication}

\subsubsection{Scripts d’assemblage}

\subsubsection{Scripts de synchronisation}

\pagebreak
