\section{Objectifs}

Deux grands buts seront ici explorés :
\begin{itmz}
\item{bénéficier d’un maximum de liberté par rapport aux outils}
\item{minimiser les risques d’altération et de perte de travail}
\end{itmz}
%⋅⋅⋅⋅⋅⋅⋅⋅⋅⋅⋅⋅⋅⋅⋅⋅⋅⋅⋅⋅⋅⋅⋅⋅⋅⋅⋅⋅⋅⋅⋅⋅⋅⋅⋅⋅⋅⋅⋅⋅⋅⋅⋅⋅⋅⋅⋅⋅⋅⋅⋅⋅⋅⋅⋅⋅⋅⋅⋅⋅⋅⋅⋅⋅⋅
\subsection{Indépendance numérique}

Trois niveaux fondamentaux sont à considérer :
\begin{itmz}
\item{les systèmes distants}
\item{les systèmes locaux}
\item{les applications}
\end{itmz}

\subsubsection{Plateformes en ligne}

De nombreuses organisations et entreprises proposent des serveurs sur Internet
permettant de stocker et synchroniser des données, voire de travailler sur des
documents directement depuis un navigateur web.

\begin{itmz}
\item{est-il possible de travailler localement en cas de coupure Internet ?}
\item{la synchronisation des données contraint-elle à un outil obligatoire ?}
\item{les données en ligne sont-elles exploitées par des tierces parties ?}
\end{itmz}

\subsubsection{Systèmes d’exploitation}

\begin{itmz}
\item{respecte-t-il la }
\item{fonctionne-t-il }
\end{itmz}

\subsubsection{Logiciels de production}

\begin{itmz}
\item{respecte-t-il la }
\item{fonctionne-t-il }
\end{itmz}
%⋅⋅⋅⋅⋅⋅⋅⋅⋅⋅⋅⋅⋅⋅⋅⋅⋅⋅⋅⋅⋅⋅⋅⋅⋅⋅⋅⋅⋅⋅⋅⋅⋅⋅⋅⋅⋅⋅⋅⋅⋅⋅⋅⋅⋅⋅⋅⋅⋅⋅⋅⋅⋅⋅⋅⋅⋅⋅⋅⋅⋅⋅⋅⋅⋅
\subsection{Fiabilité des contenus}

\subsubsection{Reproductibilité}

\subsubsection{Disponibilité}

\subsubsection{Intégrité}

\subsubsection{Authenticité}

\pagebreak
