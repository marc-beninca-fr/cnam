\section{Objectifs}

Deux grands buts seront ici explorés :
\begin{itmz}
\item{bénéficier d’un maximum de liberté par rapport aux outils}
\item{minimiser les risques d’altération et de perte de travail}
\end{itmz}
%⋅⋅⋅⋅⋅⋅⋅⋅⋅⋅⋅⋅⋅⋅⋅⋅⋅⋅⋅⋅⋅⋅⋅⋅⋅⋅⋅⋅⋅⋅⋅⋅⋅⋅⋅⋅⋅⋅⋅⋅⋅⋅⋅⋅⋅⋅⋅⋅⋅⋅⋅⋅⋅⋅⋅⋅⋅⋅⋅⋅⋅⋅⋅⋅⋅
\subsection{Indépendance numérique}

Quatre niveaux fondamentaux sont à considérer :
\begin{itmz}
\item{les systèmes distants}
\item{les systèmes locaux}
\item{les applications utilisées}
\item{les fichiers manipulés}
\end{itmz}

\subsubsection{Plateformes en ligne}

De nombreuses organisations et entreprises proposent des serveurs sur Internet
permettant de stocker et synchroniser des données, voire de travailler sur des
documents directement depuis un navigateur web.

\begin{itmz}
\item{est-il possible de travailler localement en cas de coupure Internet ?}
\item{la synchronisation des données contraint-elle à un outil obligatoire ?}
\item{les données en ligne sont-elles exploitées par des tierces parties ?}
\end{itmz}

\subsubsection{Systèmes d’exploitation}

\begin{itmz}
\item{respecte-t-il la vie privée des utilisateurs et de leurs données ?}
\item{effectue-t-il des mises à jour forceés, souvent intempestives ?}
\item{impose-t-il de la publicité et/ou d’autres distractions gênantes ?}
\item{a-t-il une obsolescence programmée forçant au rachat de matériel ?}
\end{itmz}

\subsubsection{Logiciels de production}

\begin{itmz}
\item{est-il toujours activement déveoppé et maintenu ?}
\end{itmz}

\subsubsection{Formats de fichiers}

\begin{itmz}
\item{peut-il être modifié par plusieurs logiciels ou un seul ?}
\end{itmz}
%⋅⋅⋅⋅⋅⋅⋅⋅⋅⋅⋅⋅⋅⋅⋅⋅⋅⋅⋅⋅⋅⋅⋅⋅⋅⋅⋅⋅⋅⋅⋅⋅⋅⋅⋅⋅⋅⋅⋅⋅⋅⋅⋅⋅⋅⋅⋅⋅⋅⋅⋅⋅⋅⋅⋅⋅⋅⋅⋅⋅⋅⋅⋅⋅⋅
\subsection{Fiabilité des contenus}

\subsubsection{Reproductibilité}

L’approche reproductible consiste à être capable de reconstruire un document
à partir de ses éléments constituants, au lieu de n’avoir qu’un document final.

\subsubsection{Disponibilité}

Il s’agit de l’assurance de pouvoir accéder à son document à tout moment.

\subsubsection{Intégrité}

La garantie qu’un document n’a pas été altéré
entre sa production et sa consultation.

\subsubsection{Authenticité}

La preuve qu’un document a bien été produit par son auteur déclaré.

\pagebreak
