\section{Objectifs}

Deux grands buts sont ici explorés :
\begin{enum}
\item{bénéficier d’un maximum de liberté par rapport aux outils}
\item{minimiser les risques d’altération et de perte de travail}
\end{enum}

%⋅⋅⋅⋅⋅⋅⋅⋅⋅⋅⋅⋅⋅⋅⋅⋅⋅⋅⋅⋅⋅⋅⋅⋅⋅⋅⋅⋅⋅⋅⋅⋅⋅⋅⋅⋅⋅⋅⋅⋅⋅⋅⋅⋅⋅⋅⋅⋅⋅⋅⋅⋅⋅⋅⋅⋅⋅⋅⋅⋅⋅⋅⋅⋅⋅
\hr
%⋅⋅⋅⋅⋅⋅⋅⋅⋅⋅⋅⋅⋅⋅⋅⋅⋅⋅⋅⋅⋅⋅⋅⋅⋅⋅⋅⋅⋅⋅⋅⋅⋅⋅⋅⋅⋅⋅⋅⋅⋅⋅⋅⋅⋅⋅⋅⋅⋅⋅⋅⋅⋅⋅⋅⋅⋅⋅⋅⋅⋅⋅⋅⋅⋅

\subsection{Indépendance numérique}

Quatre niveaux fondamentaux sont à considérer :
\begin{itmz}
\item{les systèmes distants}
\item{les systèmes locaux}
\item{les applications utilisées}
\item{les fichiers manipulés}
\end{itmz}

\subsubsection{Plateformes en ligne}

De nombreuses organisations et entreprises proposent des serveurs sur Internet
permettant de stocker et synchroniser des données, voire de travailler sur des
documents directement depuis un navigateur web.

\begin{itmz}
\item{est-il possible de travailler localement en cas de coupure Internet ?}
\item{la synchronisation des données contraint-elle à un outil obligatoire ?}
\item{les données en ligne sont-elles exploitées par des tierces parties ?}
\end{itmz}

\subsubsection{Systèmes d’exploitation}

Le \gls{se} :
\begin{itmz}
\item{effectue-t-il des mises à jour forceés, souvent intempestives ?}
\item{quel est son coût réel ?
    \begin{itmz}
    \item{impose-t-il de la publicité et/ou d’autres distractions gênantes ?}
    \item{respecte-t-il la vie privée des utilisateurs et de leurs données ?}
    \item{a-t-il une obsolescence programmée forçant au rachat de matériel ?}
    \end{itmz}
}
\end{itmz}

\subsubsection{Logiciels de production}

Le \gls{si} :
\begin{itmz}
\item{est-il toujours activement développé et maintenu ?}
\item{quel est son coût réel ?
    \begin{itmz}
    \item{achat périodique}
    \item{abonnement}
    \item{prix libre}
    \end{itmz}
}
\item{sous quelle licence est-il publié ?
    \begin{itmz}
    \item{propriétaire}
    \item{open source}
    \item{libre}
    \end{itmz}
}
\end{itmz}

\subsubsection{Formats de fichiers}

\begin{itmz}
\item{peut-il être modifié par plusieurs logiciels ou un seul ?}
\item{est-il documenté pour une plus grande pérennité ?
    \begin{itmz}
    \item{fermé}
    \item{obfusqué}
    \item{ouvert}
    \end{itmz}
}
\item{de quel type est-il ?
    \begin{itmz}
    \item{binaire}
    \item{texte}
    \end{itmz}
}
\end{itmz}

%⋅⋅⋅⋅⋅⋅⋅⋅⋅⋅⋅⋅⋅⋅⋅⋅⋅⋅⋅⋅⋅⋅⋅⋅⋅⋅⋅⋅⋅⋅⋅⋅⋅⋅⋅⋅⋅⋅⋅⋅⋅⋅⋅⋅⋅⋅⋅⋅⋅⋅⋅⋅⋅⋅⋅⋅⋅⋅⋅⋅⋅⋅⋅⋅⋅
\hr
%⋅⋅⋅⋅⋅⋅⋅⋅⋅⋅⋅⋅⋅⋅⋅⋅⋅⋅⋅⋅⋅⋅⋅⋅⋅⋅⋅⋅⋅⋅⋅⋅⋅⋅⋅⋅⋅⋅⋅⋅⋅⋅⋅⋅⋅⋅⋅⋅⋅⋅⋅⋅⋅⋅⋅⋅⋅⋅⋅⋅⋅⋅⋅⋅⋅

\subsection{Fiabilité des contenus}

Quatre axes rendent les contenus fiables :

\subsubsection{Disponibilité}

Il s’agit de l’assurance de pouvoir accéder à son document à tout moment.

\subsubsection{Intégrité}

La garantie qu’un document n’a pas été altéré
entre sa production et sa consultation.

\subsubsection{Authenticité}

La preuve qu’un document a bien été produit par son auteur déclaré.

\subsubsection{Reproductibilité}

L’approche reproductible consiste à être capable de reconstruire un document
à partir de ses éléments constituants, au lieu de n’avoir qu’un document final.

\begin{itmz}
\item{\gls{wysiwym} \cite{wysiwym}}
\item{\gls{wysiwyg} \cite{wysiwyg}}
\end{itmz}

\pagebreak
