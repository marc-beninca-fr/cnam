\section{Objectifs}

Ces deux grands buts permettent d’aborder plus sereinement la production :
\begin{enum}
\item{bénéficier d’un maximum de liberté par rapport aux outils}
\item{minimiser les risques d’altération et de perte de travail}
\end{enum}

%⋅⋅⋅⋅⋅⋅⋅⋅⋅⋅⋅⋅⋅⋅⋅⋅⋅⋅⋅⋅⋅⋅⋅⋅⋅⋅⋅⋅⋅⋅⋅⋅⋅⋅⋅⋅⋅⋅⋅⋅⋅⋅⋅⋅⋅⋅⋅⋅⋅⋅⋅⋅⋅⋅⋅⋅⋅⋅⋅⋅⋅⋅⋅⋅⋅
\hr
%⋅⋅⋅⋅⋅⋅⋅⋅⋅⋅⋅⋅⋅⋅⋅⋅⋅⋅⋅⋅⋅⋅⋅⋅⋅⋅⋅⋅⋅⋅⋅⋅⋅⋅⋅⋅⋅⋅⋅⋅⋅⋅⋅⋅⋅⋅⋅⋅⋅⋅⋅⋅⋅⋅⋅⋅⋅⋅⋅⋅⋅⋅⋅⋅⋅

\subsection{Indépendance numérique}

Quatre niveaux fondamentaux sont à considérer pour s’en approcher :
\begin{enum}
\item{les systèmes distants}
\item{les systèmes locaux}
\item{les applications utilisées}
\item{les fichiers manipulés}
\end{enum}

\subsubsection{Plateformes en ligne}

De nombreuses organisations et entreprises proposent des serveurs sur Internet
permettant de stocker et synchroniser des données, voire de travailler sur des
documents directement depuis un navigateur web.

Dans le cadre de l’utilisation d’une telle infrastructure,
il convient de mener une réflexion en se posant les questions suivantes :
\begin{itmz}
\item{est-il possible de continuer de travailler localement sur sa machine\\
en cas de coupure de connexion à Internet pour une durée indéterminée ?}
\item{la synchronisation des données passe-t-elle par un protocole standard,\\
permettant l’utilisation de n’importe quel logiciel compatible pour ce faire,\\
ou bien contraint-elle à l’utilisation d’un unique outil incontournable ?}
\item{les données d’utilisateurs sont-elles vendues à des tierces parties ?
    \begin{itmz}
    \item{plateformes de diffusion publicitaire en ligne}
    \item{moteurs d’apprentissage pour intelligence artificielle}
    \end{itmz}
}
\end{itmz}

\pagebreak

\subsubsection{Systèmes d’exploitation}

Également dirigés et fournis par des organisations ou des entreprises,
ils varient grandement par leurs objectifs et leur nature, et peuvent influer
de façon non négligeable sur la production de documents en leur sein.

Quelques questions à se poser au sujet d’un \gls{se} :
\begin{itmz}
\item{dispose-t-il de systèmes de fichiers modernes à base de transactions,\\
permettant de minimiser grandement le risque de corruption de données ?}
\item{permet-il aux utilisateurs de planifier les mises à jour du système,\\
en respectant leurs desideratas pour éviter toute indisponibilité de travail ?}
\item{quel est son coût réel ?
    \begin{itmz}
    \item{impose-t-il de la publicité et/ou d’autres distractions gênantes ?}
    \item{respecte-t-il la vie privée des utilisateurs et de leurs données ?}
    \item{pratique-t-il l’obsolescence programmée de ses versions successives,\\
    poussant ainsi les utilisateurs au rachat inutile de nouveau matériel ?}
    \end{itmz}
}
\end{itmz}

\subsubsection{Logiciels de production}

Outre le fait de varier aussi en fonction de l’organisation ou l’entreprise
qui le développe, un logiciel de production peut être considéré
comme un \gls{si} à part entière.

Quelques critères importants pour le choix d’un \gls{si} :
\begin{itmz}
\item{fonctionne-t-il sous le \gls{se} retenu ?}
\item{est-il toujours activement développé et maintenu ?}
\item{quel est son coût d’utilisation ?
    \begin{itmz}
    \item{\textbf{achat périodique} : à chaque sortie d’une version majeure}
    \item{\textbf{abonnement} : généralement mensuel ou annuel}
    \item{\textbf{prix libre} : via des dons facultatifs sans montant fixé}
    \end{itmz}
}
\item{sous quelle licence est-il publié ?
    \begin{itmz}
    \item{\textbf{propriétaire} :\\
    interdisant toute analyse ou modification de son fonctionnement}
    \item{\textbf{open source} :\\
    autorisant l’analyse, mais comportant des clauses restrictives}
    \item{\textbf{libre} :\\
    autorisant toute analyse, modification ou redistribution du programme}
    \end{itmz}
}
\end{itmz}

\pagebreak

\subsubsection{Formats de fichiers}

Tout \gls{si} permet de lire et écrire à partir de différents formats de fichiers.

Réflexions afin de retenir un format de fichier pour un document :
\begin{itmz}
\item{dispose-t-il d’une documentation publique détaillant sa structure,\\
permettant d’être lu et écrit par tout logiciel respectant ce standard ?}
\item{est-il documenté pour une plus grande pérennité ?
    \begin{itmz}
    \item{\textbf{fermé} : non documenté\\
    seul le logiciel de l’édtieur en permet une manipulation correcte}
    \item{\textbf{obfusqué} : partiellement documenté\\
    afin de faire dysfonctionner les logiciels tiers au fil des versions,\\
    pour inciter à utiliser le logiciel éditeur, tout en feignant l’ouverture}
    \item{\textbf{ouvert} : complètement documenté\\
    permettant à de nombreux logiciels de s’interfacer avec}
    \end{itmz}
}
\item{de quel type de format s’agit-il ?
    \begin{itmz}
    \item{\textbf{binaire} :\\
    plus performant à l’utilisation par une machine,\\
    mais nécessite des logiciels compatibles pour pouvoir les manipuler
    }
    \item{\textbf{textuel} :\\
    humainement lisible et modifiable avec n’importe quel éditeur de texte,\\
    plus facile à utiliser pour détecter et appliquer des modifications,\\
    mais prend un peu plus de temps à être interprété par une machine\\
    (écart minime, avec la puissance de calcul disponible de nos jours)
    }
    \end{itmz}
}
\end{itmz}

%⋅⋅⋅⋅⋅⋅⋅⋅⋅⋅⋅⋅⋅⋅⋅⋅⋅⋅⋅⋅⋅⋅⋅⋅⋅⋅⋅⋅⋅⋅⋅⋅⋅⋅⋅⋅⋅⋅⋅⋅⋅⋅⋅⋅⋅⋅⋅⋅⋅⋅⋅⋅⋅⋅⋅⋅⋅⋅⋅⋅⋅⋅⋅⋅⋅
\hr
%⋅⋅⋅⋅⋅⋅⋅⋅⋅⋅⋅⋅⋅⋅⋅⋅⋅⋅⋅⋅⋅⋅⋅⋅⋅⋅⋅⋅⋅⋅⋅⋅⋅⋅⋅⋅⋅⋅⋅⋅⋅⋅⋅⋅⋅⋅⋅⋅⋅⋅⋅⋅⋅⋅⋅⋅⋅⋅⋅⋅⋅⋅⋅⋅⋅

\subsection{Fiabilité des contenus}

Quatre grands axes rendent les contenus produits plus fiables :

\subsubsection{Disponibilité}

Il s’agit de l’assurance de pouvoir accéder à son document à tout moment.

Cela peut passer par deux voies :
\begin{itmz}
\item{choisir de l’hébergement garantissant une forte disponibilité}
\item{multiplier les hébergements pour être résilient à toute panne}
\end{itmz}

\pagebreak

\subsubsection{Intégrité}

La garantie qu’un document n’a pas été altéré
entre sa dernière modification et sa prochaine consultation.

Deux options peuvent être utilisées :
\begin{itmz}
\item{stocker sur des systèmes de fichiers à transaction et somme de contrôle}
\item{utiliser des outils de gestion de configuration,\\
calculant eux-mêmes des sommes de contrôle au fil des sauvegardes}
\end{itmz}

\subsubsection{Authenticité}

La preuve qu’un document a bien été produit par son auteur déclaré.

Deux mécanismes sont à mettre en œuvre, pour permettre :
\begin{itmz}
\item{à l’auteur de signer numériquement les documents qu’il produit}
\item{aux utilisateurs de vérifier eux-mêmes la validité de cette signature}
\end{itmz}

\subsubsection{Reproductibilité}

Deux philosophies de fabrication de documents s’opposent :

\begin{itmz}
\item{\textbf{\gls{wysiwyg}} : la plus courante\\
permettant de modifier un document visuellement depuis son rendu final
    \begin{itmz}
    \item{bureautique : LibreOffice \cite{lo}, MicroSoft Office \cite{mso}, WPS Office \cite{wps}}
    \end{itmz}
}
\item{\textbf{\gls{wysiwym}} : la plus pérenne\\
consistant pour un document à décrire successivement dans un fichier\\
les différents éléments à intégrer ou actions à effectuer,\\
puis à compiler un rendu final à partir de ce programme
    \begin{itmz}
    \item{moteur \TeX : \XeLaTeX, \LaTeX \cite{latex}}
    \item{GraphViz \cite{graphviz}, GnuPlot \cite{gnuplot}, Sphinx \cite{sphinx}}
    \end{itmz}
}
\end{itmz}

\pagebreak
