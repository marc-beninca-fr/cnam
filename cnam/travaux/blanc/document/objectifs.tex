\section{Objectifs}

Ces deux grands buts permettent d’aborder plus sereinement la production :
\begin{enum}
\item{bénéficier d’un maximum de liberté par rapport aux outils}
\item{minimiser les risques d’altération et de perte de travail}
\end{enum}

%⋅⋅⋅⋅⋅⋅⋅⋅⋅⋅⋅⋅⋅⋅⋅⋅⋅⋅⋅⋅⋅⋅⋅⋅⋅⋅⋅⋅⋅⋅⋅⋅⋅⋅⋅⋅⋅⋅⋅⋅⋅⋅⋅⋅⋅⋅⋅⋅⋅⋅⋅⋅⋅⋅⋅⋅⋅⋅⋅⋅⋅⋅⋅⋅⋅
\hr
%⋅⋅⋅⋅⋅⋅⋅⋅⋅⋅⋅⋅⋅⋅⋅⋅⋅⋅⋅⋅⋅⋅⋅⋅⋅⋅⋅⋅⋅⋅⋅⋅⋅⋅⋅⋅⋅⋅⋅⋅⋅⋅⋅⋅⋅⋅⋅⋅⋅⋅⋅⋅⋅⋅⋅⋅⋅⋅⋅⋅⋅⋅⋅⋅⋅

\subsection{Indépendance numérique}

Quatre niveaux fondamentaux sont à considérer pour s’en approcher :
\begin{enum}
\item{les systèmes distants}
\item{les systèmes locaux}
\item{les applications utilisées}
\item{les fichiers manipulés}
\end{enum}

\subsubsection{Plateformes en ligne}

De nombreuses organisations et entreprises proposent des serveurs sur Internet
permettant de stocker et synchroniser des données, voire de travailler sur des
documents directement depuis un navigateur web.

Dans le cadre de l’utilisation d’une telle infrastructure,
il convient de mener une réflexion en se posant les questions suivantes :
\begin{itmz}
\item{est-il possible de continuer de travailler localement sur sa machine\\
en cas de coupure de connexion à Internet pour une durée indéterminée ?}
\item{la synchronisation des données passe-t-elle par un protocole standard,\\
permettant l’utilisation de n’importe quel logiciel compatible pour ce faire,\\
ou bien contraint-elle à l’utilisation d’un unique outil incontournable ?}
\item{les données d’utilisateurs sont-elles vendues à des tierces parties ?
    \begin{itmz}
    \item{plateformes de diffusion publicitaire en ligne}
    \item{moteurs d’apprentissage pour intelligence artificielle}
    \end{itmz}
}
\end{itmz}

\pagebreak

\subsubsection{Systèmes d’exploitation}

Le \gls{se} :
\begin{itmz}
\item{effectue-t-il des mises à jour forceés, souvent intempestives ?}
\item{quel est son coût réel ?
    \begin{itmz}
    \item{impose-t-il de la publicité et/ou d’autres distractions gênantes ?}
    \item{respecte-t-il la vie privée des utilisateurs et de leurs données ?}
    \item{a-t-il une obsolescence programmée forçant au rachat de matériel ?}
    \end{itmz}
}
\end{itmz}

\subsubsection{Logiciels de production}

Le \gls{si} :
\begin{itmz}
\item{est-il toujours activement développé et maintenu ?}
\item{quel est son coût d’utilisation ?
    \begin{itmz}
    \item{achat périodique}
    \item{abonnement}
    \item{prix libre}
    \end{itmz}
}
\item{sous quelle licence est-il publié ?
    \begin{itmz}
    \item{propriétaire}
    \item{open source}
    \item{libre}
    \end{itmz}
}
\end{itmz}

\subsubsection{Formats de fichiers}

Le format du document :
\begin{itmz}
\item{peut-il être modifié par plusieurs logiciels ou bien un seul ?}
\item{est-il documenté pour une plus grande pérennité ?
    \begin{itmz}
    \item{fermé}
    \item{obfusqué}
    \item{ouvert}
    \end{itmz}
}
\item{de quel type est-il ?
    \begin{itmz}
    \item{binaire}
    \item{textuel}
    \end{itmz}
}
\end{itmz}

%⋅⋅⋅⋅⋅⋅⋅⋅⋅⋅⋅⋅⋅⋅⋅⋅⋅⋅⋅⋅⋅⋅⋅⋅⋅⋅⋅⋅⋅⋅⋅⋅⋅⋅⋅⋅⋅⋅⋅⋅⋅⋅⋅⋅⋅⋅⋅⋅⋅⋅⋅⋅⋅⋅⋅⋅⋅⋅⋅⋅⋅⋅⋅⋅⋅
\hr
%⋅⋅⋅⋅⋅⋅⋅⋅⋅⋅⋅⋅⋅⋅⋅⋅⋅⋅⋅⋅⋅⋅⋅⋅⋅⋅⋅⋅⋅⋅⋅⋅⋅⋅⋅⋅⋅⋅⋅⋅⋅⋅⋅⋅⋅⋅⋅⋅⋅⋅⋅⋅⋅⋅⋅⋅⋅⋅⋅⋅⋅⋅⋅⋅⋅

\subsection{Fiabilité des contenus}

Quatre axes rendent les contenus produits plus fiables :

\subsubsection{Disponibilité}

Il s’agit de l’assurance de pouvoir accéder à son document à tout moment.

\subsubsection{Intégrité}

La garantie qu’un document n’a pas été altéré
entre sa production et sa consultation.

\subsubsection{Authenticité}

La preuve qu’un document a bien été produit par son auteur déclaré.

\subsubsection{Reproductibilité}

L’approche reproductible consiste à être capable de reconstruire un document
à partir de ses éléments constituants, au lieu de n’avoir qu’un document final.

\begin{itmz}
\item{\gls{wysiwym} \cite{wysiwym}}
\item{\gls{wysiwyg} \cite{wysiwyg}}
\end{itmz}

\pagebreak
