\documentclass[12pt]{extarticle}
\newcommand{\import}[1]{\input{document/#1}}
%–––––––––––––––––––––––––––––––––––––––––––––––––––––––––––––––––
\import{packages}
%–––––––––––––––––––––––––––––––––––––––––––––––––––––––––––––––––
\bibliography{document}
\hypersetup{colorlinks,
citecolor=blue,
filecolor=blue,
linkcolor=blue,
urlcolor=blue,
}
\setcounter{secnumdepth}{3}
\setmainfont{DejaVu Sans}
\setmonofont{DejaVu Sans Mono}
\setlength{\parindent}{0em}
\setlength{\parskip}{1em}
\renewcommand{\baselinestretch}{1.1}
\newenvironment{itmz}{\begin{itemize}
\setlength{\itemsep}{0em}
\setlength{\parskip}{0em}
\setlength{\parsep}{0em}
}{\end{itemize}}
%\pagenumbering{gobble}
%–––––––––––––––––––––––––––––––––––––––––––––––––––––––––––––––––
\newcommand{\cnam}[0]{\begin{center}
\includegraphics[height=6em]{../cnam.png}\end{center}}
\def\fulltitle{\begin{center}\textbf{
Méthodes et outils pour\\
mieux produire documents et présentations
}\end{center}}
\def\goal{Mémoire probatoire blanc présenté en vue d’obtenir\\
UE « Information et communication pour ingénieur »\\
Spécialité :\\
Informatique, Réseaux, Systèmes et Multimédia}
\makeglossaries
\begin{document}
%–––––––––––––––––––––––––––––––––––––––––––––––––––––––––––––––––
\import{front}
%–––––––––––––––––––––––––––––––––––––––––––––––––––––––––––––––––
%\section*{Remerciements}
%\addcontentsline{toc}{section}{Remerciements}
%\pagebreak
%–––––––––––––––––––––––––––––––––––––––––––––––––––––––––––––––––
\setcounter{page}{1}
\printglossary[title=Abréviations,type=\acronymtype]

\newacronym{si}{SI}{Système d’Information}

\pagebreak
%–––––––––––––––––––––––––––––––––––––––––––––––––––––––––––––––––
%\printglossaries
%–––––––––––––––––––––––––––––––––––––––––––––––––––––––––––––––––
\printglossary[title=Glossaire]

\newglossaryentry{wysiwyg}{name=WYSIWYG,
description={What You See Is What You Get}}
\newglossaryentry{wysiwym}{name=WYSIWYM,
description={What You See Is What You Mean}}

\pagebreak
%–––––––––––––––––––––––––––––––––––––––––––––––––––––––––––––––––
\cnam

\begin{large}\fulltitle\end{large}

\renewcommand{\contentsname}{Plan}
\cftsetindents{section}{1em}{1.5em}
\cftsetindents{subsection}{2.5em}{2.5em}
\cftsetindents{subsubsection}{5em}{3.25em}
\renewcommand{\cftsecleader}{\cftdotfill{\cftdotsep}}
%\renewcommand{\cftsubsecleader}{\hfill}
\tableofcontents

\pagebreak
%–––––––––––––––––––––––––––––––––––––––––––––––––––––––––––––––––
\section{Introduction}

…

\pagebreak
%–––––––––––––––––––––––––––––––––––––––––––––––––––––––––––––––––
\section{Objectifs}

Deux grands buts seront ici explorés :
\begin{itmz}
\item{bénéficier d’un maximum de liberté par rapport aux outils}
\item{minimiser les risques d’altération et de perte de travail}
\end{itmz}

\subsection{Indépendance numérique}

Trois niveaux fondamentaux sont à considérer :
\begin{itmz}
\item{les systèmes distants}
\item{les systèmes locaux}
\item{les applications}
\end{itmz}

\subsubsection{Plateformes en ligne}

De nombreuses organisations et entreprises proposent des serveurs sur Internet
permettant de stocker et synchroniser des données, voire de travailler sur des
documents directement depuis un navigateur web.

\begin{itmz}
\item{est-il possible de travailler localement en cas de coupure Internet ?}
\item{la synchronisation des données contraint-elle à un outil obligatoire ?}
\item{les données en ligne sont-elles exploitées par des tierces parties ?}
\end{itmz}

\subsubsection{Systèmes d’exploitation}

\begin{itmz}
\item{respecte-t-il la }
\item{fonctionne-t-il }
\end{itmz}

\subsubsection{Logiciels de production}

\subsection{Fiabilité des contenus}

\subsubsection{Reproductibilité}

\subsubsection{Disponibilité}

\subsubsection{Intégrité}

\subsubsection{Authenticité}

\pagebreak
%–––––––––––––––––––––––––––––––––––––––––––––––––––––––––––––––––
\section{Moyens}

\subsection{Outils logiciels}

\subsubsection{Propriétaires}

\subsubsection{Open source}

\subsubsection{Libres}

\subsection{Formats de données}

\subsubsection{Fermés}

\subsubsection{Obfusqués}

\subsubsection{Ouverts}

\pagebreak
%⋅⋅⋅⋅⋅⋅⋅⋅⋅⋅⋅⋅⋅⋅⋅⋅⋅⋅⋅⋅⋅⋅⋅⋅⋅⋅⋅⋅⋅⋅⋅⋅⋅⋅⋅⋅⋅⋅⋅⋅⋅⋅⋅⋅⋅⋅⋅⋅⋅⋅⋅⋅⋅⋅⋅⋅⋅⋅⋅⋅⋅⋅⋅⋅⋅
\subsection{Types de formats}
Dans un \gls{si} :\\
\gls{wysiwym} \cite{wysiwym}\\
\gls{wysiwyg} \cite{wysiwyg}\\
\gls{wysiwym} est préférable à \gls{wysiwyg}

\subsubsection{Binaire}

\subsubsection{Texte}

\subsection{Rationalisation}

\subsubsection{Gestion de versions}

\subsubsection{Processus de fabrication}

\subsubsection{Logique de sauvegardes}

\subsubsection{Réplication de contenus}

\subsubsection{Signature numérique}

\pagebreak
%⋅⋅⋅⋅⋅⋅⋅⋅⋅⋅⋅⋅⋅⋅⋅⋅⋅⋅⋅⋅⋅⋅⋅⋅⋅⋅⋅⋅⋅⋅⋅⋅⋅⋅⋅⋅⋅⋅⋅⋅⋅⋅⋅⋅⋅⋅⋅⋅⋅⋅⋅⋅⋅⋅⋅⋅⋅⋅⋅⋅⋅⋅⋅⋅⋅
\subsection{Manipulation}

\subsubsection{Découpe de documents}

\subsubsection{Extraction d’images}

\subsubsection{Rotation d’éléments}

\subsubsection{Compression d’images}

\subsubsection{Résolution d’impression}

\subsubsection{Assemblage de documents}

\subsection{Automatisation}

\subsubsection{Fichiers de fabrication}

\subsubsection{Scripts d’assemblage}

\subsubsection{Scripts de synchronisation}

\pagebreak
%–––––––––––––––––––––––––––––––––––––––––––––––––––––––––––––––––
\section{Existant}

\begin{itmz}
\item{\textbf{Bureautique}\\
LibreOffice, MicroSoft Office, WPS Office\\
OpenOffice, StarOffice}
\item{\textbf{Versionnement}\\
Git, Mercurial, SubVersioN}
\item{\textbf{Fabrication}\\
GraphViz, LaTeX, PanDoc, Sphinx}
\item{\textbf{Manipulation}\\
GhostScript, ImageMagick, PDFTK, Poppler}
\item{\textbf{Automatisation}\\
BASH, Make, OpenSSH, Python, Rsync}
\item{\textbf{Authenticité}\\
GnuPG, OpenPGP}
\end{itmz}

\pagebreak
%–––––––––––––––––––––––––––––––––––––––––––––––––––––––––––––––––
\section{Conclusion}

…

\pagebreak
%–––––––––––––––––––––––––––––––––––––––––––––––––––––––––––––––––
\printbibliography[heading=bibintoc,title=Bibliographie]

\pagebreak
%–––––––––––––––––––––––––––––––––––––––––––––––––––––––––––––––––
\appendix
%–––––––––––––––––––––––––––––––––––––––––––––––––––––––––––––––––
\section{Fabrication de ce document}

\definecolor{bg}{rgb}{0.9,0.9,0.9}
\definecolor{cm}{rgb}{0.1,0.6,0.1}
\definecolor{kw}{rgb}{0.8,0.1,0.1}
\definecolor{str}{rgb}{0.1,0.1,0.8}

\lstset{
basicstyle=\ttfamily,
numbers=left,
backgroundcolor=\color{bg},
commentstyle=\color{cm},
keywordstyle=\color{kw},
stringstyle=\color{str},
}

\subsection{Script de construction}

Programme Python :

\lstinputlisting[language=Python]{build.py}

\pagebreak
%⋅⋅⋅⋅⋅⋅⋅⋅⋅⋅⋅⋅⋅⋅⋅⋅⋅⋅⋅⋅⋅⋅⋅⋅⋅⋅⋅⋅⋅⋅⋅⋅⋅⋅⋅⋅⋅⋅⋅⋅⋅⋅⋅⋅⋅⋅⋅⋅⋅⋅⋅⋅⋅⋅⋅⋅⋅⋅⋅⋅⋅⋅⋅⋅⋅
\subsection{Fichier principal}

Programme \XeLaTeX :

\lstinputlisting[language={[LaTeX]TeX},otherkeywords={
hypersetup,setmainfont,setmonofont,setlength,
setmainlanguage,
maketitle,tableofcontents,subsection,appendix,
gls,makeglossaries,newacronym,newglossaryentry,printglossary,
definecolor
}]{document.tex}

\pagebreak
%⋅⋅⋅⋅⋅⋅⋅⋅⋅⋅⋅⋅⋅⋅⋅⋅⋅⋅⋅⋅⋅⋅⋅⋅⋅⋅⋅⋅⋅⋅⋅⋅⋅⋅⋅⋅⋅⋅⋅⋅⋅⋅⋅⋅⋅⋅⋅⋅⋅⋅⋅⋅⋅⋅⋅⋅⋅⋅⋅⋅⋅⋅⋅⋅⋅
\subsection{Base bibliographique}

Fichier Biber :

\lstdefinelanguage{bib}{morestring=[b]",
keywords={author,howpublished,misc,note,title,url}}
\lstinputlisting[language={bib}]{document.bib}

\pagebreak
%–––––––––––––––––––––––––––––––––––––––––––––––––––––––––––––––––
\import{back}
%–––––––––––––––––––––––––––––––––––––––––––––––––––––––––––––––––
\end{document}
