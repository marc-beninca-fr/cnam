\documentclass[12pt]{extarticle}
%–––––––––––––––––––––––––––––––––––––––––––––––––––––––––––––––––
\usepackage{polyglossia}
\setmainlanguage{french}
\usepackage{extsizes}
\usepackage{fontspec}
\usepackage[a4paper,portrait,
bmargin=20mm,lmargin=20mm,rmargin=20mm,tmargin=20mm]{geometry}
\usepackage{metalogo}
\usepackage{tocloft}
\usepackage{wallpaper}
\usepackage[acronym,toc]{glossaries}
\usepackage{listings}
\usepackage[sorting=none,backend=biber]{biblatex}
\usepackage{hyperref}
%–––––––––––––––––––––––––––––––––––––––––––––––––––––––––––––––––
\bibliography{document}
\hypersetup{colorlinks,
citecolor=blue,
filecolor=blue,
linkcolor=blue,
urlcolor=blue,
}
\setcounter{secnumdepth}{2}
\setmainfont{DejaVu Sans}
\setmonofont{DejaVu Sans Mono}
\setlength{\parindent}{0em}
\setlength{\parskip}{1em}
\renewcommand{\baselinestretch}{1}
%\pagenumbering{gobble}
%–––––––––––––––––––––––––––––––––––––––––––––––––––––––––––––––––
\newcommand{\cnam}[0]{\ThisURCornerWallPaper{.2}{../cnam.png}}
\makeglossaries
\begin{document}
%–––––––––––––––––––––––––––––––––––––––––––––––––––––––––––––––––
\cnam

\title{\textbf{Produire de\\
meilleurs documents}\\
 \\
concepts et outils}
\author{Marc BENINCA}
\date{02 / 05 / 2020}
\maketitle
\thispagestyle{empty}

\pagebreak
%–––––––––––––––––––––––––––––––––––––––––––––––––––––––––––––––––
%\section*{Remerciements}
%\addcontentsline{toc}{section}{Remerciements}
%\pagebreak
%–––––––––––––––––––––––––––––––––––––––––––––––––––––––––––––––––
\setcounter{page}{1}
\printglossary[title=Abréviations,type=\acronymtype]

\newacronym{si}{SI}{Système d’Information}

\pagebreak
%–––––––––––––––––––––––––––––––––––––––––––––––––––––––––––––––––
%\printglossaries
%–––––––––––––––––––––––––––––––––––––––––––––––––––––––––––––––––
\printglossary[title=Glossaire]

\newglossaryentry{wysiwyg}{name=WYSIWYG,
description={What You See Is What You Get}}
\newglossaryentry{wysiwym}{name=WYSIWYM,
description={What You See Is What You Mean}}

\pagebreak
%–––––––––––––––––––––––––––––––––––––––––––––––––––––––––––––––––
\cnam

\textbf{Produire de meilleurs documents}\\
concepts et outils

\renewcommand{\contentsname}{Sommaire}
\cftsetindents{section}{1em}{1.5em}
\cftsetindents{subsection}{2.5em}{2.5em}
\renewcommand{\cftsecleader}{\hfill}
%\renewcommand{\cftsubsecleader}{\hfill}
\tableofcontents

\pagebreak
%–––––––––––––––––––––––––––––––––––––––––––––––––––––––––––––––––
\section{Introduction}

\pagebreak
%–––––––––––––––––––––––––––––––––––––––––––––––––––––––––––––––––
\section{Objectifs}

\subsection{Indépendance numérique}
\begin{itemize}
\item{logiciels de production}
\item{systèmes d’exploitation}
\item{plateformes d’hébergement}
\end{itemize}

\subsection{Fiabilité des contenus}
\begin{itemize}
\item{reproductibilité}
\item{disponibilité}
\item{intégrité}
\item{authenticité}
\end{itemize}

\pagebreak
%–––––––––––––––––––––––––––––––––––––––––––––––––––––––––––––––––
\section{Moyens}

\subsection{Outils logiciels}
\begin{itemize}
\item{propriétaires}
\item{open source}
\item{libres}
\end{itemize}

\subsection{Formats de données}
\begin{itemize}
\item{fermés}
\item{obfusqués}
\item{ouverts}
\end{itemize}

\pagebreak
%⋅⋅⋅⋅⋅⋅⋅⋅⋅⋅⋅⋅⋅⋅⋅⋅⋅⋅⋅⋅⋅⋅⋅⋅⋅⋅⋅⋅⋅⋅⋅⋅⋅⋅⋅⋅⋅⋅⋅⋅⋅⋅⋅⋅⋅⋅⋅⋅⋅⋅⋅⋅⋅⋅⋅⋅⋅⋅⋅⋅⋅⋅⋅⋅⋅
\subsection{Types de formats}
Dans un \gls{si} :\\
\gls{wysiwym} \cite{wysiwym}\\
\gls{wysiwyg} \cite{wysiwyg}\\
\gls{wysiwym} est préférable à \gls{wysiwyg}

\begin{itemize}
\item{binaire}
\item{texte}
\end{itemize}

\subsection{Rationalisation}
\begin{itemize}
\item{gestion de versions}
\item{processus de fabrication}
\item{logique de sauvegardes}
\item{réplication de contenus}
\item{signature numérique}
\end{itemize}

\pagebreak
%⋅⋅⋅⋅⋅⋅⋅⋅⋅⋅⋅⋅⋅⋅⋅⋅⋅⋅⋅⋅⋅⋅⋅⋅⋅⋅⋅⋅⋅⋅⋅⋅⋅⋅⋅⋅⋅⋅⋅⋅⋅⋅⋅⋅⋅⋅⋅⋅⋅⋅⋅⋅⋅⋅⋅⋅⋅⋅⋅⋅⋅⋅⋅⋅⋅
\subsection{Manipulation}
\begin{itemize}
\item{découpe de documents}
\item{extraction d’images}
\item{rotation d’éléments}
\item{compression d’images}
\item{résolution d’impression}
\item{assemblage de documents}
\end{itemize}

\subsection{Automatisation}
\begin{itemize}
\item{fichiers de fabrication}
\item{scripts d’assemblage}
\item{scripts de synchronisation}
\end{itemize}

\pagebreak
%–––––––––––––––––––––––––––––––––––––––––––––––––––––––––––––––––
\section{Existant}

\begin{itemize}
\item{\textbf{Bureautique}\\
LibreOffice, MicroSoft Office, WPS Office\\
OpenOffice, StarOffice}
\item{\textbf{Versionnement}\\
Git, Mercurial, SubVersioN}
\item{\textbf{Fabrication}\\
GraphViz, LaTeX, PanDoc, Sphinx}
\item{\textbf{Manipulation}\\
GhostScript, ImageMagick, PDFTK, Poppler}
\item{\textbf{Automatisation}\\
BASH, Make, OpenSSH, Python, Rsync}
\item{\textbf{Authenticité}\\
GnuPG, OpenPGP}
\end{itemize}

\pagebreak
%–––––––––––––––––––––––––––––––––––––––––––––––––––––––––––––––––
\section{Conclusion}

\pagebreak
%–––––––––––––––––––––––––––––––––––––––––––––––––––––––––––––––––
\printbibliography[heading=bibintoc,title=Bibliographie]

\pagebreak
%–––––––––––––––––––––––––––––––––––––––––––––––––––––––––––––––––
\appendix
%–––––––––––––––––––––––––––––––––––––––––––––––––––––––––––––––––
\section{Fabrication de ce document}

\definecolor{bg}{rgb}{0.9,0.9,0.9}
\definecolor{cm}{rgb}{0.1,0.6,0.1}
\definecolor{kw}{rgb}{0.8,0.1,0.1}
\definecolor{str}{rgb}{0.1,0.1,0.8}

\lstset{
basicstyle=\ttfamily,
numbers=left,
backgroundcolor=\color{bg},
commentstyle=\color{cm},
keywordstyle=\color{kw},
stringstyle=\color{str},
}

\subsection{Script de construction}

Programme Python :

\lstinputlisting[language=Python]{build.py}

\pagebreak
%⋅⋅⋅⋅⋅⋅⋅⋅⋅⋅⋅⋅⋅⋅⋅⋅⋅⋅⋅⋅⋅⋅⋅⋅⋅⋅⋅⋅⋅⋅⋅⋅⋅⋅⋅⋅⋅⋅⋅⋅⋅⋅⋅⋅⋅⋅⋅⋅⋅⋅⋅⋅⋅⋅⋅⋅⋅⋅⋅⋅⋅⋅⋅⋅⋅
\subsection{Fichier principal}

Programme \XeLaTeX :

\lstinputlisting[language={[LaTeX]TeX},otherkeywords={
hypersetup,setmainfont,setmonofont,setlength,
setmainlanguage,
maketitle,tableofcontents,subsection,appendix,
gls,makeglossaries,newacronym,newglossaryentry,printglossary,
definecolor
}]{document.tex}

\pagebreak
%⋅⋅⋅⋅⋅⋅⋅⋅⋅⋅⋅⋅⋅⋅⋅⋅⋅⋅⋅⋅⋅⋅⋅⋅⋅⋅⋅⋅⋅⋅⋅⋅⋅⋅⋅⋅⋅⋅⋅⋅⋅⋅⋅⋅⋅⋅⋅⋅⋅⋅⋅⋅⋅⋅⋅⋅⋅⋅⋅⋅⋅⋅⋅⋅⋅
\subsection{Base bibliographique}

Fichier Biber :

\lstdefinelanguage{bib}{morestring=[b]",
keywords={author,howpublished,misc,note,title,url}}
\lstinputlisting[language={bib}]{document.bib}

\pagebreak
%–––––––––––––––––––––––––––––––––––––––––––––––––––––––––––––––––
\cnam
\newcommand{\hr}[0]{\rule{\textwidth}{1pt}}

\title

Mémoire présenté en vue d’obtenir\\
UE « Information et communication pour ingénieur »\\
Spécialité : Informatique\\
Bordeaux, 2020

\hr

Résumé

\hr

Summary

\pagebreak
%–––––––––––––––––––––––––––––––––––––––––––––––––––––––––––––––––
\end{document}
%–––––––––––––––––––––––––––––––––––––––––––––––––––––––––––––––––
