% TODO répertoire metafiles
% TODO bibliographie
% TODO décaler la numérotation des pages
\documentclass[12pt]{extarticle}
%–––––––––––––––––––––––––––––––––––––––––––––––––––––––––––––––––––––––––––––––
\usepackage{polyglossia}
\setmainlanguage{french}
\usepackage{extsizes}
\usepackage{fontspec}
\usepackage[a4paper,portrait,
bmargin=20mm,lmargin=20mm,rmargin=20mm,tmargin=20mm]{geometry}
\usepackage{metalogo}
\usepackage{tocloft}
\usepackage{wallpaper}
\usepackage[acronym]{glossaries}
\usepackage{hyperref}
%–––––––––––––––––––––––––––––––––––––––––––––––––––––––––––––––––––––––––––––––
\hypersetup{colorlinks,
citecolor=blue,
filecolor=blue,
linkcolor=blue,
urlcolor=blue
}
\setcounter{secnumdepth}{2}
\setmainfont{DejaVu Sans}
\setlength{\parindent}{0em}
\setlength{\parskip}{1em}
\renewcommand{\baselinestretch}{1}
%\pagenumbering{gobble}
%–––––––––––––––––––––––––––––––––––––––––––––––––––––––––––––––––––––––––––––––
\newcommand{\cnam}[0]{\ThisURCornerWallPaper{.2}{../cnam.png}}
\makeglossaries
\begin{document}
%–––––––––––––––––––––––––––––––––––––––––––––––––––––––––––––––––––––––––––––––
\cnam

\title{\textbf{Produire de\\
meilleurs documents}\\
 \\
concepts et outils}
\author{Marc BENINCA}
\date{30 / 04 / 2020}
\maketitle

\pagebreak
%–––––––––––––––––––––––––––––––––––––––––––––––––––––––––––––––––––––––––––––––
\newacronym{si}{SI}{Système d’Information}

\renewcommand{\acronymname}{Abréviations}
\printglossary[type=\acronymtype]

\pagebreak
%–––––––––––––––––––––––––––––––––––––––––––––––––––––––––––––––––––––––––––––––
%\printglossaries
%–––––––––––––––––––––––––––––––––––––––––––––––––––––––––––––––––––––––––––––––
\newglossaryentry{wysiwyg}{name=WYSIWYG, description={What You See Is What You Get}}
\newglossaryentry{wysiwym}{name=WYSIWYM, description={What You See Is What You Mean}}

\printglossary

\pagebreak
%–––––––––––––––––––––––––––––––––––––––––––––––––––––––––––––––––––––––––––––––
\cnam

\textbf{Produire de meilleurs documents}\\
concepts et outils

\renewcommand{\contentsname}{Sommaire}
\cftsetindents{section}{1em}{1.5em}
\cftsetindents{subsection}{2.5em}{2.5em}
\renewcommand{\cftsecleader}{\hfill}
%\renewcommand{\cftsubsecleader}{\hfill}
\tableofcontents

\pagebreak
%–––––––––––––––––––––––––––––––––––––––––––––––––––––––––––––––––––––––––––––––
\section{Introduction}

\pagebreak
%–––––––––––––––––––––––––––––––––––––––––––––––––––––––––––––––––––––––––––––––
\section{Objectifs}

\subsection{Indépendance numérique}
\begin{itemize}
\item{logiciels de production}
\item{systèmes d’exploitation}
\item{plateformes d’hébergement}
\end{itemize}

\subsection{Fiabilité des contenus}
\begin{itemize}
\item{reproductibilité}
\item{disponibilité}
\item{intégrité}
\item{authenticité}
\end{itemize}

\pagebreak
%–––––––––––––––––––––––––––––––––––––––––––––––––––––––––––––––––––––––––––––––
\section{Moyens}

\subsection{Outils logiciels}
\begin{itemize}
\item{propriétaires}
\item{open source}
\item{libres}
\end{itemize}

\subsection{Formats de données}
\begin{itemize}
\item{fermés}
\item{obfusqués}
\item{ouverts}
\end{itemize}

\pagebreak
%⋅⋅⋅⋅⋅⋅⋅⋅⋅⋅⋅⋅⋅⋅⋅⋅⋅⋅⋅⋅⋅⋅⋅⋅⋅⋅⋅⋅⋅⋅⋅⋅⋅⋅⋅⋅⋅⋅⋅⋅⋅⋅⋅⋅⋅⋅⋅⋅⋅⋅⋅⋅⋅⋅⋅⋅⋅⋅⋅⋅⋅⋅⋅⋅⋅⋅⋅⋅⋅⋅⋅⋅⋅⋅⋅⋅⋅⋅⋅
\subsection{Types de formats}
Dans un \gls{si} :\\
\gls{wysiwyg}\\
\gls{wysiwym}\\
\gls{wysiwym} est préférable à \gls{wysiwyg}

\begin{itemize}
\item{binaire}
\item{texte}
\end{itemize}

\subsection{Rationalisation}
\begin{itemize}
\item{gestion de versions}
\item{processus de fabrication}
\item{logique de sauvegardes}
\item{réplication de contenus}
\item{signature numérique}
\end{itemize}

\pagebreak
%⋅⋅⋅⋅⋅⋅⋅⋅⋅⋅⋅⋅⋅⋅⋅⋅⋅⋅⋅⋅⋅⋅⋅⋅⋅⋅⋅⋅⋅⋅⋅⋅⋅⋅⋅⋅⋅⋅⋅⋅⋅⋅⋅⋅⋅⋅⋅⋅⋅⋅⋅⋅⋅⋅⋅⋅⋅⋅⋅⋅⋅⋅⋅⋅⋅⋅⋅⋅⋅⋅⋅⋅⋅⋅⋅⋅⋅⋅⋅
\subsection{Manipulation}
\begin{itemize}
\item{découpe de documents}
\item{extraction d’images}
\item{rotation d’éléments}
\item{compression d’images}
\item{résolution d’impression}
\item{assemblage de documents}
\end{itemize}

\subsection{Automatisation}
\begin{itemize}
\item{fichiers de fabrication}
\item{scripts d’assemblage}
\item{scripts de synchronisation}
\end{itemize}

\pagebreak
%–––––––––––––––––––––––––––––––––––––––––––––––––––––––––––––––––––––––––––––––
\section{Existant}

\begin{itemize}
\item{\textbf{Bureautique}\\
LibreOffice, MicroSoft Office, WPS Office\\
OpenOffice, StarOffice}
\item{\textbf{Versionnement}\\
Git, Mercurial, SubVersioN}
\item{\textbf{Fabrication}\\
GraphViz, LaTeX, PanDoc, Sphinx}
\item{\textbf{Manipulation}\\
GhostScript, ImageMagick, PDFTK, Poppler}
\item{\textbf{Automatisation}\\
BASH, Make, OpenSSH, Python, Rsync}
\item{\textbf{Authenticité}\\
GnuPG, OpenPGP}
\end{itemize}

\pagebreak
%–––––––––––––––––––––––––––––––––––––––––––––––––––––––––––––––––––––––––––––––
\section{Conclusion}

\pagebreak
%–––––––––––––––––––––––––––––––––––––––––––––––––––––––––––––––––––––––––––––––
\cnam
\newcommand{\hr}[0]{\rule{\textwidth}{1pt}}

\title

Mémoire présenté en vue d’obtenir\\
UE « Information et communication pour ingénieur »\\
Spécialité : Informatique\\
Bordeaux, 2020

\hr

Résumé

\hr

Summary

\pagebreak
%–––––––––––––––––––––––––––––––––––––––––––––––––––––––––––––––––––––––––––––––
\end{document}
%–––––––––––––––––––––––––––––––––––––––––––––––––––––––––––––––––––––––––––––––
