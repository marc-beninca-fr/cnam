\section{Exemple}

L’exemple suivant, en 2 parties, illustre bien le propos.

\textbf{Linéaire} \cite{ex-linear}

…

\fig{}{Bon ou mauvais en \gls{ml},\\
d’après les notes de Mathématiquess et Statistiques}
{24em}{ex-linear-plot}

…

\fig{}{Séparation à Vaste Marge très nette}
{24em}{ex-linear-svm}

…

\fig{}{Anomalies dans le \gls{ds}, violations de marge}
{24em}{ex-linear-out}

…

\fig{}{Différentes séparations à marge souple, variation de C}
{28em}{ex-linear-soft}

…

\pagebreak

\textbf{Non linéaire} \cite{ex-nonlinear}

…

\fig{}{\Gls{ds} inséparable de façon linéaire}
{24em}{ex-nonlinear-plot}

…

{\LARGE
$X_{1}=x_{1}^{2}$

$X_{2}=x_{2}^{2}$

$X_{3}=\sqrt{2} × x_{1} × x_{2}$
}

…

\fig{}{\Gls{hpp} séparateur linéaire dans le nouvel espace dimensionnel}
{24em}{ex-nonlinear-linear}

…

\fig{}{Fonction de décision non linéaire dans l’espace d’origine}
{24em}{ex-nonlinear-svm}

…

\fig{}{Projection de la marge de séparation dans l’espace d’origine}
{24em}{ex-nonlinear-sv}

…

\pagebreak
