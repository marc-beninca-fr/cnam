\section{Principes}

L’approche \gls{svm} est un ensemble de méthodes supervisées utilisant :
\begin{enum}
\item{un \gls{ds} d’apprentissage pour entraîner l’algorithme,\\
et qui fait donc office de superviseur}
\item{un \gls{ds} de test pour vérifier sa pertinence}
\end{enum}

Cette approche se révèle appropriée dans de nombreux cas d’utilisation :
\begin{itmz}
\item{filtrage d’email, courriel légitime ou pourriel (phishing, spam)}
\item{classification d’images, quel que soit le \gls{si}}
\item{détection de \gls{sgn} dans des fichiers multimédias}
\item{quantification de granularité dans des textures}
\item{reconnaissance de caractères dans des images}
\item{classification d’expressions faciales dans des images}
\item{reconnaissance vocale dans des échantillons sonores}
\item{classification de protéines}
\item{établissement de diagnostics médicaux}
\item{classification de documents en différentes catégories}
\end{itmz}

En fonction du type de problèmes, deux types de résolution :
\begin{itmz}
\item{\textbf{régression}}
\item{\textbf{classification} (la plus courante)}
\end{itmz}

En fonction des \glspl{ds}, deux types d’approches mathématiques :
\begin{itmz}
\item{\textbf{linéaire} : la plus simple}
\item{\textbf{non linéaire} : faisant appel à des \glspl{kf}}
\end{itmz}

Quatre paramètres permettent d’affiner le modèle :
\begin{itmz}
\item{\textbf{noyau} : linéaire, \gls{rbf}, polynominal ou \gls{sigmoid}}
\item{\textbf{degré} : aide à trouver un \gls{hpp} séparateur en contexte polynominal,
faisant rapidement augmenter le temps nécessaire à l’entraînement}
\item{\textbf{gamma} : pour les \glspl{hpp} non linéaires}
\item{\textbf{C} : pénalité augmentant la distance des données prises en compte,\\
au risque d’engendrer un surentraînement si trop importante}
\end{itmz}

\pagebreak

\subsection{Régression}

…

\subsubsection{Régression linéaire}

\bifig{}{Régression linéaire \cite{homl-linear}}
{16em}{regression_linear_left}{regression_linear_right}

…

\subsubsection{Régression non linéaire}

\bifig{}{Régression non linéaire \cite{homl-nonlinear}}
{16em}{regression_nonlinear_left}{regression_nonlinear_right}

…

\subsection{Classification}

…

\subsubsection{Classification linéaire}

…

\bifig{}{Vaste marge \cite{homl-large-scale}}
{9em}{margin_large_left}{margin_large_right}

…

\bifig{}{ \cite{homl-large-scale}}
{10em}{margin_scale_left}{margin_scale_right}

…

\bifig{}{ \cite{homl-hard-few}}
{9em}{margin_hard_left}{margin_hard_right}

…

\bifig{}{ \cite{homl-hard-few}}
{9em}{margin_few_left}{margin_few_right}

…

\subsubsection{Classification non linéaire}

…

\bifig{}{Séparabilité linéaire \cite{homl-nonlinear-linear}}
{14em}{nonlinear_linear_left}{nonlinear_linear_right}

…

\fig{}{ \cite{homl-feat-poly}}
{9em}{features_polynomial}

…

\bifig{}{\Gls{kf} polynominale \cite{homl-poly}}
{14em}{kernel_polynomial_left}{kernel_polynomial_right}

…

\bifig{}{ \cite{homl-feat-simi}}
{14em}{features_similar_left}{features_similar_right}

…

\bifig{}{\Gls{kf} gaussien \gls{rbf} \cite{homl-rbf}}
{26em}{kernel_rbf_left}{kernel_rbf_right}

…

\cite{svm}

\pagebreak
