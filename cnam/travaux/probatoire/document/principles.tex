\section{Principes}

L’approche \gls{svm} est une méthode supervisée de classification d’éléments
utilisant :
\begin{enum}
\item{un \gls{dataset} d’apprentissage pour entraîner l’algorithme,\\
et qui fait donc office de superviseur}
\item{un \gls{dataset} de test pour vérifier sa pertinence}
\end{enum}

Cette approche se révèle appropriée dans de nombreux cas d’utilisation :
\begin{itmz}
\item{classification d’images, quel que soit le \gls{si}}
\item{classification de protéines, dans le champ médical}
\item{filtrage d’email, courriel légitime ou pourriel (phishing, spam)}
\end{itmz}

En fonction du type de problèmes à résoudre,
deux types de résolution sont disponibles :
\begin{itmz}
\item{régression linéaire}
\item{régression non linéaire}
\item{classification linéaire}
\item{classification non linéaire}
\end{itmz}

\subsection{Régression}

…

\subsubsection{Régression linéaire}

…

\subsubsection{Régression non linéaire}

…

\subsection{Classification}

…

\subsubsection{Classification linéaire}

…

\subsubsection{Classification non linéaire}

…

\cite{csd}

\pagebreak
