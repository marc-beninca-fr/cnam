\printglossary[title=Abréviations,type=\acronymtype]

\newacronym{anova}{ANOVA}{ANalysis Of VAriance}
\newacronym{hbos}{HBOS}{Histogram Based Outlier Score}
\newacronym{irm}{IRM}{Imagerie par Résonnance Magnétique}
\newacronym{nn}{NN}{Nearest Neighbors}
\newacronym{rbf}{RBF}{Radial Basis Function}
\newacronym{sgbd}{SGBD}{Systèmes de Gestion de Bases de Données}
\newacronym{si}{SI}{Systèmes d’Information}
\newacronym{svc}{SVC}{Support Vector Classification}
\newacronym{svm}{SVM}{Support Vector Machine}
\newacronym{svr}{SVR}{Support Vector Regression}
\newacronym{tic}{TIC}{Technologies d’Information et de Communication}

\pagebreak

\printglossary[title=Glossaire]

\newglossaryentry{bd}{
name={big data},
description={données massives}
}
\newglossaryentry{clustering}{
name={clustering},
description={regroupement d’éléments en sous-ensembles caractéristiques}
}
\newglossaryentry{ds}{
name={dataset},
plural={datasets},
description={ensemble de données}
}
\newglossaryentry{gs}{
name={grid search},
description={algorithme d’affinage d’hyperparamètres par grille de valeurs}
}
\newglossaryentry{kf}{
name={fonction noyau},
plural={fonctions noyau},
description={transformation non linéaire permettant une séparation linéaire}
}
\newglossaryentry{hpp}{
name={hyperplan},
plural={hyperplans},
description={sous-espace en n−1 dimesions d’un espace en n dimensions}
}
\newglossaryentry{kt}{
name={kernel trick},
description={astuce du noyau pour éviter des calculs plus complexes}
}
\newglossaryentry{ml}{
name={machine learning},
description={apprentissage machine automatique}
}
\newglossaryentry{sgn}{
name={stéganographie},
description={dissimulation d’informations dans un plus grand ensemble}
}
\newglossaryentry{sigmoid}{
name={sigmoïde},
description={courbe à double asymptôte 0 et 1, centrée sur 1÷2}
}
\newglossaryentry{sv}{
name={support vector},
plural={support vectors},
description={sous-ensemble de données d’entraînement}
}

\pagebreak

%\printglossaries
