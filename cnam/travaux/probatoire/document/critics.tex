\section{Critiques}


\subsection{Avantages}


\begin{itmz}

\item{le modèle \gls{svm} est considéré comme plutôt stable, car\\
de petits changements dans les données impactent peu les \glspl{hpp}}

\end{itmz}\begin{itmz}

\item{ce même modèle peut être utilisé à la fois pour\\
résoudre des problèmes de régression et de classification}

\end{itmz}\begin{itmz}

\item{est souvent une meilleure alternative aux réseaux de neurones,\\
surtout dans le cas de \glspl{ds} de taille réduite}

\end{itmz}\begin{itmz}

\item{très précis avec une marge de séparation nette}

\end{itmz}\begin{itmz}

\item{peut capturer des relations complexes dans des \glspl{ds},\\
même sans connaissance préalable de ces dernières}

\end{itmz}\begin{itmz}

\item{fonctionne avec un grand nombres d’espaces dimensionnels}

\end{itmz}\begin{itmz}

\item{reste pertinent même avec plus de dimensions que d’éléments}

\end{itmz}\begin{itmz}

\item{capacité de régularisation pour rester générique,\\
afin d’éviter les risques de surentraînement}

\end{itmz}\begin{itmz}

\item{économise la mémoire nécessaire aux fonctions de décision,\\
en ne leur faisant traiter que des \glspl{sv}}

\end{itmz}\begin{itmz}

\item{permet de gérer efficacement des données non linéaires,\\
par l’intermédiaire de différentes \glspl{kf}}

\end{itmz}\begin{itmz}

\item{de nombreuses \glspl{kf} existent pour les fonctions de décision}

\end{itmz}\begin{itmz}

\item{il est possible de combiner plusieurs \glspl{kf},\\
pour pouvoir travailler avec des \glspl{hpp} plus complexes}

\end{itmz}\begin{itmz}

\item{pendant longtemps considéré inadapté aux très grands \glspl{ds},\\
de nouveaux algorithmes ont été mis à contribution depuis \cite{large-scale}}

\end{itmz}


\subsection{Inconvénients}


\begin{itmz}

\item{le choix d’une \gls{kf} appropriée n’est pas évident,\\
et peut facilement mener à un trop grand nombre de \glspl{sv}}

\end{itmz}\begin{itmz}

\item{la mémoire nécessaire augmente avec le nombre de \glspl{sv},\\
car ces derniers doivent y être intégralement stockés}

\end{itmz}\begin{itmz}

\item{difficile d’identifier les bonnes valeurs de paramètres,\\
sans passer par des techniques comme \gls{gs}}

\end{itmz}\begin{itmz}

\item{le temps d’entraînement augmente avec le nombre d’éléments}

\end{itmz}\begin{itmz}

\item{les calculs de probabilité de justesse sont très coûteux,\\
nécessitant une validation croisée en plusieurs étapes}

\end{itmz}\begin{itmz}

\item{les modèles sont difficilement interprétables par des humains,\\
contrairement par exemple aux arbres de décisions}

\end{itmz}\begin{itmz}

\item{effet boîte noire en cas de compréhension insuffisante\\
des différents outils mathématiques sous-jacents}

\end{itmz}


\subsection{Limitations}


\begin{itmz}

\item{fonctionne mal quand des classes se recouvrent, car\\
plusieurs paramètres varient, contrairement à d’autres méthodes}

\end{itmz}\begin{itmz}

\item{une normalisation préalable des données est nécessaire,\\
pour que les fonctions objectifs soient plus pertinentes}

\end{itmz}\begin{itmz}

\item{les problèmes multi-classes sont encore\\
un champ de recherche à part entière}

\end{itmz}


\pagebreak
