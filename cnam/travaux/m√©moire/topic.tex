\documentclass[10pt]{article}

\usepackage{fontspec}
\usepackage[a4paper,portrait,
bmargin=10mm,lmargin=15mm,rmargin=15mm,tmargin=10mm]{geometry}

\pagenumbering{gobble}
\setlength{\parindent}{0em}
\setlength{\parskip}{0em}
\setmainfont{DejaVu Sans}

\newcommand{\hr}{\rule{\textwidth}{1pt}}

\newenvironment{itmz}{\begin{itemize}
\setlength{\itemsep}{0em}
}{\end{itemize}}

\begin{document}

CNAM / UAMM91 \hfill Mémoire ingénieur / Sujet \hfill IRSM (CYC9104A)

Marc Beninca \hfill Systèmes d’exploitation autonomes incrémentaux \hfill 2020 → 2021

\hr

\section{Problématiques d’utilisation des systèmes d’exploitation}

\subsection{Partitions systèmes avec accès en écriture}

\subsubsection{Système de fichiers sans gestion d’instantanés}

\begin{itmz}
\item{\textbf{avantages} : instantanéité des modifications faites aux fichiers systèmes}
\item{\textbf{inconvénients} : nécessiter de régulièrement faire et tester des sauvegardes}
\end{itmz}

\subsubsection{Système de fichiers avec gestion d’instantanés}

\begin{itmz}
\item{\textbf{avantages} : permet de sauvegarder et restaurer un état des fichiers à tout moment}
\item{\textbf{inconvénients} : peu d’implémentations, pas encore utilisés par défaut}
\end{itmz}

\subsection{Images autonomes avec accès en lecture seule}

\subsubsection{Amorçage sans gestion de persistance}

\begin{itmz}
\item{\textbf{avantages} : démarrage d’un système autonome dans un état figé au préalable}
\item{\textbf{inconvénients} : perte au redémarrage de toutes modifications faites aux fichiers systèmes}
\end{itmz}

\subsubsection{Amorçage avec gestion de persistance}

\begin{itmz}
\item{\textbf{avantages} : conservation des modifications faites aux fichiers systèmes}
\item{\textbf{inconvénients} : pas de séparation entre persistance de fichiers systèmes et de données}
\end{itmz}

\section{Proposition : fonctionnement autonome incrémental}

\begin{itmz}
\item{\textbf{avantages} : restauration très rapide, chaque mise à jour devient une sauvegarde}
\item{\textbf{inconvénients} : maintenance exhaustive si effectuée sans assistances}
\end{itmz}

\subsection{Fabrication}

\subsubsection{Miroirs de dépôts officiels distribution et éditeurs}

\begin{itmz}
\item{synchronisation locale pour accès rapide, stable et hors-ligne : apt-mirror, debmirror, ftpsync}
\item{vérification d’intégrité avant utilisation des paquets synchronisés}
\end{itmz}

\subsubsection{Construction d’un système de fichiers}

\begin{itmz}
\item{prise en compte du type de machine cible : physique, virtuelle, conteneur}
\item{détermination des paquets logiciels à installer et à configurer}
\item{transformation en système d’exploitation « à la volée »}
\end{itmz}

\subsubsection{Encapsulation dans un fichier image}

\begin{itmz}
\item{choix d’un format adapté au montage en lecture seule : squashfs}
\item{algorithmes de compression disponibles : gzip, lzma, lzo, lz4, xz, zstd}
\item{niveau supplémentaire d’encapsulation avec un format hybride : iso}
\end{itmz}

\subsubsection{Sécurité}

\begin{itmz}
\item{intégrité par calcul de sommes de contrôle : sha512sum}
\item{authenticité via signature des images produites : gpg}
\end{itmz}

\subsection{Amorçage}

\begin{itmz}
\item{utilisation d’un chargeur de démarrage signé : grub, secure boot}
\item{création d’un menu de démarrage à choix multiple d’images}
\item{vérification d’authenticité et d’intégrité des images}
\item{chargement d’image(s) en mémoire vive : complet, partiel}
\end{itmz}

\subsection{Mise à niveau}

\begin{itmz}
\item{système d’exploitation en cours d’exécution}
\item{nouvelle image pour le prochain redémarrage}
\end{itmz}

\section{Scripts d’automatisation potentiels}

\begin{itmz}
\item{synchronisation locale et vérification d’intégrité, de façon parallélisée}
\item{construction de systèmes de fichiers à partir de profils versionnés}
\item{génération à la volée de menu de démarrage}
\end{itmz}

\hr

\appendix

\section{Contexte professionnel}

Étant engagé de carrière, depuis 19 ans envers le Ministère des Armées, je travaille :
\begin{itmz}
\item{depuis 2001 pour le corps des sous-officiers de l’Armée de l’Air et de l’Espace}
\item{depuis 2016 sous la Direction Interarmées des Réseaux d’Infrastructure et Systèmes d’Information}
\end{itmz}

\subsection{Restrictions}

\begin{itmz}
\item{interdiction statutaire d’exercer simultanément une autre activité professionnelle}
\item{pas de droits d’administration sur le gestionnaire de l’infrastructure virtuelle distante}
\item{stations de travail locales sans virtualisation, administrées par une entité extérieure}
\item{ordinateurs portables réquisitionnés pour le personnel exerçant son activité en télétravail}
\end{itmz}

\subsection{Propriété intellectuelle}

\begin{itmz}
\item{les développements effectués sur le temps de travail deviennent la propriété du Ministère}
\item{des scripts d’automatisation auraient bien plus de portée s’ils étaient publiés sous licence libre}
\end{itmz}

\section{Cadre de production du mémoire ingénieur}

Compte tenu des contraintes énoncées ci-dessus, est-il possible :
\begin{itmz}
\item{de construire mon mémoire ingénieur Hors Temps de Travail ?}
\end{itmz}

\end{document}
