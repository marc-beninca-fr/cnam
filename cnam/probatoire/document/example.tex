\section{Exemples}

Les 2 cas suivants,
linéaire \cite{ex-linear} et non linéaire \cite{ex-nonlinear},
illustrent bien le propos.

\subsection{Linéaire}

Une classification est voulue pour pouvoir déterminer si un étudiant
sera bon ou non en \gls{ml}, à partir des 2 notes qu’il a obtenu
aux examens de Mathématiques et de Statistiques.

\fig{}{\Gls{ds} très bien distribué}
{16em}{ex-linear-plot}

Avec un tel \gls{ds}, l’identification des \glspl{sv} est aisée,
et une marge la plus large possible est facilement applicable.

\fig{}{Séparation à Vaste Marge très nette}
{16em}{ex-linear-svm}

\pagebreak

\fig{}{Anomalies dans le \gls{ds}, violations de marge}
{16em}{ex-linear-out}

Pour un \gls{ds} avec aberrations, faire varier une marge souple permet
de trouver un compromis entre généralisation et spécialisation.

\fig{}{Différentes séparations à marge souple, variation de C}
{32em}{ex-linear-soft}

\pagebreak

\subsection{Non linéaire}

\fig{}{\Gls{ds} inséparable de façon linéaire}
{16em}{ex-nonlinear-plot}

Création d’un nouvel espace dimensionnel avec des transformations
non linéaires des variables d’origine.

{\large
$X_{1}=x_{1}^{2}$ ; $X_{2}=x_{2}^{2}$ ; $X_{3}=\sqrt{2} × x_{1} × x_{2}$
}

Ce nouvel espace permet de trouver une marge et un \gls{hpp} séparateur.

\fig{}{\Gls{hpp} séparateur linéaire dans le nouvel espace dimensionnel}
{24em}{ex-nonlinear-linear}

\pagebreak

\fig{}{Projection de la marge de séparation dans l’espace d’origine}
{24em}{ex-nonlinear-sv}

Le retour à l’espace de départ permet ainsi marge et séparation non linéaires.

\fig{}{Fonction de décision finale non linéaire dans l’espace d’origine}
{24em}{ex-nonlinear-svm}

\pagebreak
